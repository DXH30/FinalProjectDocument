\documentclass[./skripsi.tex]{subfiles}
\begin{document}
\chapter{Hasil Penelitian}
\section{Deskripsi Hasil Penelitian}
% Buat dalam bentuk tabel dan diperjelas
% CNN LSTM Training Testing
% Hasil Prediksi
\par Hasil penelitian terdiri dari data hasil training CNN, LSTM, dan data hasil testing pada snortIDS. Data Hasil dari LSTM dan CNN merupakan data hasil Training, Testing, dan hasil prediksi dari model \textit{Neural Network}.
\par Data hasil yang diperoleh dari proses \textit{filtering} dengan metode CNN adalah akurasi dan loss data per paket yang dihitung berdasarkan konvolusi dari nilai \textit{raw packet}. Sedangkan data hasil yang diperoleh dari proses \textit{profiling} dengan metode LSTM adalah data keseluruhan paket header yang diperhitungkan secara sekuensial atau regresi. Data hasil pada model LSTM merupakan nilai pendekatan dari trafik asli.
\par Sebelum mengolah data menjadi input \textit{neural network}, perlu dilakukan konversi data dari data mentah menjadi data yang dapat dibaca. Dataset yang digunakan diambil file pcap dan file executable nya.
\par Data yang dijadikan input pada sisi CNN adalah seluruh data yang masuk pada host. Sedangkan data yang dijadikan input pada sisi LSTM adalah seluruh data header baik keluar maupun masuk dari host. Proses yang dilakukan untuk memperoleh data hasil adalah dengan menggunakan SnortIDS mengirimkan 10 parameter header dengan payloadnya.
\par Pada sisi \textit{webservice} dengan menggunakan \textit{flask} akan didefinisikan buffer pada setiap data yang masuk berdasarkan sumber alamat IP nya. Buffer untuk IP ini terbagi menjadi 3 yakni, buffer transmit, receive, dan transceive. Buffer transmit, dan receive yang berkaitan akan digabungkan untuk mendeteksi intrusi pada sisi packet header dengan LSTM. Dan buffer transmit pada setiap IP akan digunakan untuk mendeteksi intrusi pada sisi packet payload dengan CNN.
\clearpage
\section{Hasil Training Data Neural Network}
\subsection{Model CNN}
\begin{figure}%[H]
    \centering
    \includegraphics[width=0.7\textwidth]{public/assets/img/CNNModel.png}
    \caption{Model CNN}
    \label{fig:model_cnn}
\end{figure}
\par Berdasarkan gambar \ref{fig:model_cnn}, pada model CNN terdapat layer Embedding digunakan untuk mengekstraksi fitur dari data input dengan ukuran 2xMTU atau 3000 Byte. Layer selanjutnya adalah layer Reshape yang mengubah ukuran input dari 3000x50 menjadi 3000x50x1 dengan kata lain menginklusi matriks dari matriks 2 dimensi menjadi matriks 3 dimensi dengan ketebalan 1.
\par Layer selanjutnya merupakan layer konvolusi, pada konvolusi pertama dan kedua outputnya di normalisasi dengan BatchNormalization, dan pada Layers konvolusi terakhir dilakukan MaxPooling untuk meniadakan interpretasi yang akan mengurangi rate pendeteksian. Setelah semua konvolusi selesai matriks dinormalisasi lagi sebelum masuk ke layer selanjutnya.
\par Layer selanjutnya merupakan layer untuk deteksi. Pada layer Flatten semua matriks pada layer konvolusi terakhir diubah menjadi vektor. Kemudian setelah itu masuk ke layer Dense atau untuk aktivasi pertama lalu masuk ke Dropout untuk mengurangi neuron irrelevan dari Dense sebelumnya. Kemudian keluaran dari Dropout Layer menjadi input bagi Layer Dense terakhir yang berukuran vektor 2 bit, atau memiliki 2 kelas. Dari kedua kelas ini hanya kategori [0 1] yang dilakukan training data.
\subsection{Model LSTM}
\begin{figure}%[H]
    \centering
    \includegraphics[width=0.5\textwidth]{public/assets/img/LSTMModel.png}
    \caption{Model LSTM}
    \label{fig:lstm_model}
\end{figure}
\par Pada Gambar \ref{fig:lstm_model} model LSTM hanya terdapat layer LSTM sebagai pembaca vektor header dari packet. Dropout untuk menghilangkan neuron yang irrelevan. Layer Activation dimana hasil keluaran Layer sebelumnya diaktivasi dengan menggunakan fungsi aktivasi \textit{softmax} dan Dense dengan ukuran klaster 1, yakni memiliki skalar untuk pendeteksiannya.
\par Untuk proses pendeteksian dengan sentiment analysis neuron Dense berjumlah 1, sedangkan untuk prediksi multivariable neuron Dense berjumlah 10, dimana neuron ini merupakan jumlah parameter pada header packet.
\subsection{Hasil Training Data LSTM}
\par Sistem training data LSTM menggunakan 2 metode yang berbeda yakni \textit{Sentiment Analysis} dan \textit{Multivariate Prediction}. Hasil \textit{training} kedua metode ini memiliki perbedaan dimana pada \textit{Sentiment Analysis} dapat memprediksi apabila variasi dan distribusi variasi pada targetnya merata. Sedangkan pada \textit{Multivariate Prediction} hasil training memiliki karakteristik konvergen yang tinggi karena memperhitungkan dan memprediksi data secara keseluruhan.

\subsubsection{Hasil training Svchosta Botnet dengan Sentiment Analysis LSTM}
\par Untuk Hasil training svchosta dengan sentiment analysis memiliki data sebagai berikut :
\begin{figure}%[H]
    \centering
    \includegraphics[width=0.7\textwidth]{public/assets/img/lstms_svchosta_loss.png}
    \caption{Loss model LSTM svchosta Sentiment Analysis}
    \label{fig:lstms_svchosta_loss}
\end{figure}
\begin{figure}%[H]
	\centering
    \includegraphics[width=0.7\textwidth]{public/assets/img/lstms_svchosta_acc.png}
    \caption{Akurasi model LSTM svchosta Sentiment Analysis}
    \label{fig:lstms_svchosta_loss}
\end{figure}

\par Dapat diamati pada grafik kisaran akurasi Loss dari model svchosta test ada di antara 0.016 dan 0.015, sedangkan kisaran akurasi Loss dari model svchosta train ada di antara 0.011 dan 0.012. Hasil akurasi diperoleh dapat diamati juga kisaran train berada di sekitar 0.989, dan kisaran test berada di sekitar 0.985.

\begin{figure}%[H]
    \centering
    \includegraphics[width=0.7\textwidth]{public/assets/img/lstms_svchosta_pred.png}
    \caption{Prediksi model LSTM svchosta Sentiment Analysis}
    \label{fig:lstms_svchosta_pred}
\end{figure}

\par Dari grafik \ref{fig:lstms_svchosta_pred} dapat diamati bentuk prediksi \textit{LSTM Sentiment Analysis}. Tampak variasi trafik label yang diperoleh di prediksi hanya bernilai 0, sedangkan beberapa label lain tidak terdeteksi sama sekali. Hal ini dikarenakan jarangnya trafik malware yang ada pada jaringan. Pemberian angka dilakukan pada setiap label yang berbeda. Nilai 0 diperoleh dikarenakan label yang dianalisis bukan merupakan label dari botnet.

% Tabel Hasil Sentiment
\begin{table}%[H]
\centering
\caption{Tabel Hasil LSTMS Svchosta}
\begin{tabelkeras}
\hline
0  &  0.015402 &  0.98474 &                 0.015402 &                       1.0 &  0.011336 &  0.98888 &             0.011336 &                   1.0 \\
5  &  0.015288 &  0.98474 &                 0.015288 &                       1.0 &  0.011274 &  0.98888 &             0.011274 &                   1.0 \\
10 &  0.015347 &  0.98474 &                 0.015347 &                       1.0 &  0.011353 &  0.98888 &             0.011353 &                   1.0 \\
15 &  0.015331 &  0.98474 &                 0.015331 &                       1.0 &  0.011340 &  0.98888 &             0.011340 &                   1.0 \\
20 &  0.015695 &  0.98474 &                 0.015695 &                       1.0 &  0.011318 &  0.98888 &             0.011318 &                   1.0 \\
25 &  0.015413 &  0.98474 &                 0.015413 &                       1.0 &  0.011345 &  0.98888 &             0.011345 &                   1.0 \\
30 &  0.015288 &  0.98474 &                 0.015288 &                       1.0 &  0.011383 &  0.98888 &             0.011383 &                   1.0 \\
35 &  0.015349 &  0.98474 &                 0.015349 &                       1.0 &  0.011274 &  0.98888 &             0.011274 &                   1.0 \\
40 &  0.015405 &  0.98474 &                 0.015405 &                       1.0 &  0.011274 &  0.98888 &             0.011274 &                   1.0 \\
45 &  0.015570 &  0.98474 &                 0.015570 &                       1.0 &  0.011395 &  0.98888 &             0.011395 &                   1.0 \\
\hline
\end{tabelkeras}
\label{table:lstms_svchosta}
\end{table}

\par Berdasarkan tabel \ref{table:lstms_svchosta} dapat dilihat tidak terjadi konvergensi pada (Value Loss) VL, (Loss) L, maupun parameter (A) Akurasi. Nilai yang diperoleh di sisi Loss bernilai 0.011 dan akurasi 0.989. Tidak terjadi perubahan setiap epoch. Beberapa parameter terjadi sedikit fluktuasi pada sisi prediksi.

\subsubsection{Hasil Training Neris Botnet dengan Sentiment Analysis LSTM}
\par Untuk Hasil training Neris dengan sentiment analysis memiliki data sebagai berikut :

\begin{figure}%[H]
    \centering
    \includegraphics[width=0.7\textwidth]{public/assets/img/lstms_neris_loss.png}
    \caption{Loss model LSTM Neris Sentiment Analysis}
    \label{fig:lstms_neris_loss}
\end{figure}
\begin{figure}%[H]
	\centering
    \includegraphics[width=0.7\textwidth] {public/assets/img/lstms_neris_acc.png}
    \caption{Akurasi model LSTM neris Sentiment Analysis}
    \label{fig:lstms_neris_acc}
\end{figure}

\par Berdasarkan grafik \ref{fig:lstms_neris_acc} dan \ref{fig:ig:lstms_neris_loss} dapat dilihat bahwa pola grafik tidak menunjukkan perubahan, hal ini berarti tidak terjadi proses pembelajaran. Tidak adanya proses pembelajaran ini disebabkan karena jumlah data yang bernilai malicious pada \textit{header} sangat kecil atau tidak seimbang dengan data yang bernilai normal. Hal ini menyebabkan data dapat diabaikan.

\begin{figure}%[H]
    \centering
    \includegraphics[width=0.7\textwidth]{public/assets/img/lstms_neris_pred.png}
    \caption{Prediksi model LSTM neris Sentiment Analysis}
    \label{fig:lstms_neris_pred}
\end{figure}

\par Pada grafik \ref{fig:lstms_neris_pred} dapat damati bentuk prediksi \textit{LSTM Sentiment Analysis}. Dimana pada sisi loss model dari hasil testing berada di kisaran 0.0108 sampai 0.0112, sedangkan loss di hasil training berada di kisaran 0.0118, dan 0.0116. Pada sisi akurasi model dari hasil testing berada di kisaran 0.9893, sedangkan dari hasil training berada di kisaran 0.9885. Tidak terjadi perubahan pada kedua grafik ini dikarenakan tidak terjadiny a pembelajaran pada model ini.

% Tabel Hasil Sentiment
\begin{table}%[H]
\centering
\caption{Tabel Hasil LSTMS Neris}
\begin{tabelkeras}
\hline
0  &  0.010916 &  0.989313 &                 0.010916 &                       1.0 &  0.011702 &  0.988516 &             0.011702 &                   1.0 \\
5  &  0.010751 &  0.989313 &                 0.010751 &                       1.0 &  0.011737 &  0.988516 &             0.011737 &                   1.0 \\
10 &  0.010809 &  0.989313 &                 0.010809 &                       1.0 &  0.011778 &  0.988516 &             0.011778 &                   1.0 \\
15 &  0.010946 &  0.989313 &                 0.010946 &                       1.0 &  0.011678 &  0.988516 &             0.011678 &                   1.0 \\
20 &  0.011027 &  0.989313 &                 0.011027 &                       1.0 &  0.011685 &  0.988516 &             0.011685 &                   1.0 \\
25 &  0.010838 &  0.989313 &                 0.010838 &                       1.0 &  0.011664 &  0.988516 &             0.011664 &                   1.0 \\
30 &  0.010762 &  0.989313 &                 0.010762 &                       1.0 &  0.011652 &  0.988516 &             0.011652 &                   1.0 \\
35 &  0.011100 &  0.989313 &                 0.011100 &                       1.0 &  0.011623 &  0.988516 &             0.011623 &                   1.0 \\
40 &  0.011066 &  0.989313 &                 0.011066 &                       1.0 &  0.011616 &  0.988516 &             0.011616 &                   1.0 \\
45 &  0.010992 &  0.989313 &                 0.010992 &                       1.0 &  0.011733 &  0.988516 &             0.011733 &                   1.0 \\
\hline
\end{tabelkeras}
\label{table:lstms_neris}
\end{table}

\par Dapat dilihat pada tabel \ref{table:lstms_neris} tidak terjadi konvergensi pada (Value Loss) VL, (Loss) L, maupun parameter (A) Akurasi. Hal ini disebabkan karena tidak terjadi pembelajaran data dikarenakan dataset memiliki label yang tidak seimbang.

\subsubsection{Hasil Training Rbot Botnet dengan Sentiment Analysis LSTM}
\par Untuk Hasil training Rbot dengan sentiment analysis memiliki data sebagai berikut :

\begin{figure}%[H]
    \centering
    \includegraphics[width=0.7\textwidth]{public/assets/img/lstms_rbot_loss.png}
    \caption{Loss model LSTM Rbot Sentiment Analysis}
    \label{fig:lstms_rbot_loss}
\end{figure}
\begin{figure}%[H]
    \centering
    \includegraphics[width=0.7\textwidth]{public/assets/img/lstms_rbot_acc.png}
    \caption{Akurasi model LSTM rbot Sentiment Analysis}
    \label{fig:lstms_rbot_acc}
\end{figure}

\par Berdasarkan dari analisis grafik \ref{fig:lstms_rbot_loss} dan \ref{fig:lstms_rbot_acc} dapat diamati bahwa tidak adanya pembelajaran. Dapat diamati bahwa tidak ada perubahan nilai baik pada akurasi maupun pada loss dari model. Hal ini disebabkan karena data yang bernilai malicious atau intrusi pada \textit{header} sangat kecil sehingga dapat diabaikan.

\begin{figure}%[H]
    \centering
    \includegraphics[width=0.7\textwidth]{public/assets/img/lstms_rbot_pred.png}
    \caption{Prediksi model LSTM rbot Sentiment Analysis}
    \label{fig:lstms_rbot_pred}
\end{figure}

\par Dapat diamati bentuk prediksi dari grafik \ref{fig:lstms_rbot_pred} yang menunjukkan hasil prediksi \textit{LSTM Sentiment Analysis}. Karena saat training tidak terjadi konvergensi, sehingga data prediksi yang diperoleh tidak menunjukkan kecocokan antara hasil prediksi dengan data yang seharusnya.

% Tabel Hasil Sentiment

\begin{table}%[H]
\centering
\caption{Tabel Hasil LSTMS RBot}
\begin{tabelkeras}
\hline
0  &  0.025622 &  0.974672 &                 0.025622 &                       1.0 &  0.015825 &  0.984372 &             0.015825 &                   1.0 \\
5  &  0.025387 &  0.974672 &                 0.025387 &                       1.0 &  0.015732 &  0.984372 &             0.015732 &                   1.0 \\
10 &  0.025461 &  0.974672 &                 0.025461 &                       1.0 &  0.015732 &  0.984372 &             0.015732 &                   1.0 \\
15 &  0.025463 &  0.974672 &                 0.025463 &                       1.0 &  0.015892 &  0.984372 &             0.015892 &                   1.0 \\
20 &  0.025491 &  0.974672 &                 0.025491 &                       1.0 &  0.015925 &  0.984372 &             0.015925 &                   1.0 \\
25 &  0.025443 &  0.974672 &                 0.025443 &                       1.0 &  0.015815 &  0.984372 &             0.015815 &                   1.0 \\
30 &  0.025374 &  0.974672 &                 0.025374 &                       1.0 &  0.015709 &  0.984372 &             0.015709 &                   1.0 \\
35 &  0.025417 &  0.974672 &                 0.025417 &                       1.0 &  0.015730 &  0.984372 &             0.015730 &                   1.0 \\
40 &  0.025413 &  0.974672 &                 0.025413 &                       1.0 &  0.015744 &  0.984372 &             0.015744 &                   1.0 \\
45 &  0.025428 &  0.974672 &                 0.025428 &                       1.0 &  0.015821 &  0.984372 &             0.015821 &                   1.0 \\
\hline
\end{tabelkeras}
\label{table:lstms_rbot}
\end{table}

\par Tabel \ref{table:lstms_rbot} menunjukkan tidak adanya konvergensi pada (Value Loss) VL, (Loss) L, maupun parameter (A) Akurasi. Hal ini disebabkan karena tidak terjadi pembelajaran data dikarenakan dataset memiliki label yang tidak seimbang atau data malicious yang sangat jarang.

% Mulai Multivariate Prediction
\subsubsection{Hasil Training svchosta Botnet dengan Multivariate Prediction LSTM}
\par Untuk Hasil training svchosta dengan multivariate prediction memiliki data sebagai berikut :

\begin{figure}%[H]
	\centering
    \includegraphics[width=0.7\textwidth]{public/assets/img/lstmm_svchosta_loss.png}
    \caption{Loss model LSTM svchosta Sentiment Analysis}
    \label{fig:lstmm_svchosta_loss}
\end{figure}

\begin{figure}%[H]
	\centering
    \includegraphics[width=0.7\textwidth]{public/assets/img/lstmm_svchosta_acc.png}
    \caption{Akurasi model LSTM svchosta Sentiment Analysis}
    \label{fig:lstmm_svchosta_acc}
\end{figure}

\par Dapat dilihat bahwa grafik menjelaskan bahwa hasil training konvergen dan memiliki loss terendah sebesar 0.03 pada sisi testing dan 0.05 pada sisi training. Dapat dilihat juga dari grafik yang mengalami pemusatan pada satu sumbu y baik pada grafik \textit{Loss} maupun pada grafik \textit{Accuracy} berbeda dari metode sebelumnya.

\par Parameter yang diukur memiliki karakteristik prediksi sebagai berikut :
\begin{enumerate}
    \item Proto
    \begin{figure}%[H]
        \centering
        \includegraphics[width=0.7\textwidth]{public/assets/img/lstmm_svchosta_pred1.png}
        \caption{Grafik Prediksi Proto Svchosta}
        \label{fig:lstmm_svchosta_pred1}
    \end{figure}
    
    \par Bagian \ref{fig:lstmm_svchosta_pred1} ini merupakan salah satu bagian dari paket jaringan yang mendefinisikan jenis protokol jaringan yang digunakan. Dapat dilihat dari grafik bahwa sebagian besar hasil prediksi berada di atas tapi memiliki pola yang mirip dengan trafik asli.
    \par Titik hijau merupakan data asli sedangkan titik merah merupakan data hasil prediksi model. Proses perubahan data proto menjadi grafik dilakukan dengan metode tokenisasi, dimana setiap data berbeda dilabeli dengan angka dan di sesuaikan dengan jumlah variasi datanya.
    
    \item SrcAddr
    \begin{figure}%[H]
        \centering
        \includegraphics[width=0.7\textwidth] {public/assets/img/lstmm_svchosta_pred2.png}
        \caption{Grafik Prediksi SrcAddr Svchosta}
        \label{fig:lstmm_svchosta_pred2}
    \end{figure}
    
    \par Gambar \ref{fig:lstmm_svchosta_pred2} merupakan bagian IP header dari paket jaringan yang mendefinisikan alamat IP sumber. Dapat dilihat dari grafik bahwa sebagian besar hasil prediksi hanya mencakup nilai rendah saja, sedangkan trafik asli tersebar.
    \par Dapat diamati pada gambar bahwa titik hijau menunjukkan data asli sedangkan titik merah menunjukkan data prediksi. Proses pemetaan ip ke grafik juga sama dengan pada proto, menggunakan metode tokenisasi dimana data unik akan diberi label nilai dari rentang 0 sampai 1 berdasarkan jumlah variasinya.
    
    \item DstAddr
    \begin{figure}%[H]
        \centering
        \includegraphics[width=0.7\textwidth]{public/assets/img/lstmm_svchosta_pred3.png}
        \caption{Grafik Prediksi DstAddr Svchosta}
        \label{fig:lstmm_svchosta_pred3}
    \end{figure}
    
    \par Pada gambar \ref{fig:lstmm_svchosta_pred3} merupakan bagian IP header dari paket jaringan yang mendefinsikan alamat IP tujuan. Dapat dilihat dari grafik bahwa sebagian besar hasil prediksi dapat mencakup keseluruhan trafik asli.
    \par Titik merah menunjukkan data prediksi, sedangkan titik hijau menunjukkan data trafik asli. Proses pemetaan juga sama seperti sebelumnya dengan metode tokenisasi.
    
    \item Sport
    \begin{figure}%[H]
        \centering
        \includegraphics[width=0.7\textwidth]{public/assets/img/lstmm_svchosta_pred4.png}
        \caption{Grafik Prediksi Sport Svchosta}
        \label{fig:lstmm_svchosta_pred4}
    \end{figure}
    
    \par Pada gambar \ref{fig:lstmm_svchosta_pred4} menjelaskan persebaran data dari IP header yang mendefinisikan port sumber. Dapat dilihat dari grafik hasil pendeteksian port sumber mencakup keseluruhan, walaupun ada beberapa bagian yang tidak dicakup terutama dibagian nilai tertinggi.
    \par Titik hijau menunjukkan data trafik asli sedangkan titik merah menunjukkan data trafik hasil prediksi. Proses pemetaan menjadi grafik juga menggunakan metode tokenisasi.
    
    \item Dport
    \begin{figure}%[H]
        \centering
        \includegraphics[width=0.7\textwidth]{public/assets/img/lstmm_svchosta_pred5.png}
        \caption{Grafik Prediksi Dport Svchosta}
        \label{fig:lstmm_svchosta_pred5}
    \end{figure}
    
    \par Gambar \ref{fig:lstmm_svchosta_pred5} ini merupakan bagian dari IP header yang mendefinisikan port tujuan. Dapat dilihat dari grafik hasil pendeteksian port tujuan bahwa hasil pendeteksian mencakup keseluruhan trafik port tujuan, sementara beberapa bagian sekitar angka port tertinggi dan terendah tidak ada di hasil prediksi.
    \par Titik hijau dan titik merah menunjukkan data asli dan hasil prediksi sama seperti sebelumnya. Proses pemetaan juga menggunakan metode tokenisasi.
    
    \item Dir
    \begin{figure}%[H]
        \centering
        \includegraphics[width=0.7\textwidth]{public/assets/img/lstmm_svchosta_pred6.png}
        \caption{Grafik Prediksi Dir Svchosta}
        \label{fig:lstmm_svchosta_pred6}
    \end{figure}
    
    \par Bagian \ref{fig:lstmm_svchosta_pred6} ini merupakan arah yang menunjukan arah transmisi dari paket baik itu multicast, maupun unicast. Dapat dilihat dari grafik hasil pendeteksian dapat mencakup hampir keseluruhan trafik kecuali kategori 1.0.
    \par Titik merah menunjukkan prediksi dan titik hijau menunjukkan data asli. Pemetaan pada data ini juga menggunakan metode tokenisasi.
    
    \item TotPkts
    \begin{figure}%[H]
        \centering
        \includegraphics[width=0.7\textwidth]{public/assets/img/lstmm_svchosta_pred7.png}
        \caption{Grafik Prediksi TotPkts Svchosta}
        \label{fig:lstmm_svchosta_pred7}
    \end{figure}
    
    \par Grafik \ref{fig:lstmm_svchosta_pred7} ini menjelaskan bagian dari paket header yang menjelaskan jumlah paket dalam suatu ethernet frame yang dikirim per paket. Dari grafik dapat dilihat variasi total paket hasil prediksi sebagian besar dapat terdeteksi sementara bagian variasi kelas ukuran disekitar 1.0 dan 0.9 tidak terdeteksi.
    \par Berdasarkan grafik \ref{fig:lstmm_svchosta_pred7} dapat diamati titik merah yang merupakan data hasil prediksi dan titik hijau yang merupakan data asli. Proses pemetaan juga melalui metode tokenisasi.
    
    \item TotBytes
    \begin{figure}%[H]
        \centering
        \includegraphics[width=0.7\textwidth]{public/assets/img/lstmm_svchosta_pred8.png}
        \caption{Grafik Prediksi TotBytes Svchosta}
        \label{fig:lstmm_svchosta_pred8}
    \end{figure}
    
    \par Bagian ini merupakan bagian dari paket header yang menjelaskan jumlah bytes dalam satu ethernet frame yang dikirim per paket. Dari grafik dapat dilihat variasinya menyerupai totpkts, sedangkan hasil prediksi memiliki cakupan yang sangat besar menutupi sebagian besar kelas total byte rendah.
    
    \item SrcBytes
    \begin{figure}%[H]
        \centering
        \includegraphics[width=0.7\textwidth]{public/assets/img/lstmm_svchosta_pred9.png}
        \caption{Grafik Prediksi SrcBytes Svchosta}
        \label{fig:lstmm_svchosta_pred9}
    \end{figure}
    
    \par Pada bagian \ref{fig:lstmm_svchosta_9} ini merupakan bagian dari paket header yang menjelaskan ukuran byte source. Dapat dilihat trafik yang terdeteksi mencakup sebagian besar kecuali sisi kelas terendah atau bawah yang bernilai 0.1.
    \par Titik hijau menjelaskan data asli dan titik merah data prediksi. Proses pemetaan sama seperti sebelumnya yakni dengan metode tokenisasi.
    
    \item Label
    \begin{figure}%[H]
        \centering
        \includegraphics[width=0.7\textwidth]{public/assets/img/lstmm_svchosta_pred10.png}
        \caption{Grafik Prediksi Label Svchosta}
        \label{fig:lstmm_svchosta_pred10}
    \end{figure}
    
    \par Bagian ini merupakan label dari dataset yang sudah di angka kan. Dapat dilihat dari grafik bahwa sebagian besar terdeteksi sedangkan kelas label tertinggi atau sekitar 0.9 sampai 1.0 tidak terdeteksi.
    \par Titik merah menunjukkan data prediksi sedangkan titik hijau menunjukkan data asli. Proses pemetaan juga melalui proses tokenisasi
\end{enumerate}

\par Berdasarakan grafik keseluruhan dari hasil prediksi kesepuluh parameter \textit{LSTM Multivariate Prediction}, dapat dilihat bahwa pada beberapa parameter prediksi terdapat beberapa parmaeter yang berhasil di prediksi dan beberapa parameter pula yang tidak dapat terdeteksi.

% Tabel Hasil
\begin{table}%[H]
\centering
\caption{Tabel Hasil LSTMM Svc}
\begin{tabelkeras}
\hline
0  &  0.076987 &  0.764155 &                 0.076987 &                  0.764155 &  0.128025 &  0.506444 &             0.128025 &              0.506444 \\
5  &  0.034513 &  0.373282 &                 0.034513 &                  0.373282 &  0.048499 &  0.708092 &             0.048499 &              0.708092 \\
10 &  0.033781 &  0.380621 &                 0.033781 &                  0.380621 &  0.045702 &  0.702732 &             0.045702 &              0.702732 \\
15 &  0.032518 &  0.378274 &                 0.032518 &                  0.378274 &  0.044133 &  0.698664 &             0.044133 &              0.698664 \\
20 &  0.031653 &  0.376445 &                 0.031653 &                  0.376445 &  0.042929 &  0.696064 &             0.042929 &              0.696064 \\
25 &  0.031494 &  0.376694 &                 0.031494 &                  0.376694 &  0.042119 &  0.696388 &             0.042119 &              0.696388 \\
30 &  0.029571 &  0.376736 &                 0.029571 &                  0.376736 &  0.041559 &  0.694804 &             0.041559 &              0.694804 \\
35 &  0.029802 &  0.375568 &                 0.029802 &                  0.375568 &  0.041111 &  0.696844 &             0.041111 &              0.696844 \\
40 &  0.029912 &  0.377669 &                 0.029912 &                  0.377669 &  0.040611 &  0.695416 &             0.040611 &              0.695416 \\
45 &  0.028958 &  0.376280 &                 0.028958 &                  0.376280 &  0.040329 &  0.694196 &             0.040329 &              0.694196 \\
\hline
\end{tabelkeras}
\label{table:lstmm_svchosta}
\end{table}

\par Berdasarkan tabel \ref{table:lstmm_svchosta} diatas dapat diamati bahwa penurunan Loss terjadi dan peningkatan terjadi. Pada sisi akurasi dapat dilihat bahwa akurasi tertinggi yang dicapai model adalah 0.7. Hal ini dikarenakan beberapa parameter tidak mempengaruhi satu sama lain, namun berdasarkan keseluruhan hasil prediksi yang diperoleh dapat digunakan untuk melakukan profiling.

\par Metode pemetaan dan keterangan titik juga berlaku pada data hasil training neris botnet dan rbot botnet.

\subsubsection{Hasil Training Neris Botnet dengan Multivariate Prediction LSTM}

\begin{figure}%[H]
    \centering
    \includegraphics[width=0.7\textwidth]{public/assets/img/lstmm_neris_loss.png}
    \caption{Loss model LSTM neris Multivariate Prediction}
    \label{fig:lstmm_neris_loss}
\end{figure}

\begin{figure}%[H]
    \centering
    \includegraphics[width=0.7\textwidth]{public/assets/img/lstmm_neris_acc.png}
    \caption{Akurasi model LSTM neris Multivariate Prediction}
    \label{fig:lstmm_neris_acc}
\end{figure}

\par Berdasarkan dari grafik diatas dapat diamati bahwa hasil training konvergen. Loss terendah yang dicapai pada sisi testing berada disekitar 0.03, sedangkan loss terendah dicapai pada sisi training adalah sekitar 0.05. 
\par Dapat dilihat dari grafik yang mengalami pemusatan pada satu sumbu y baik pada grafik \textit{Loss} maupun pada grafik \textit{Accuracy}. Dapat dilihat juga dari grafik akurasi pada hasil testing memiliki fluktuasi akurasi tapi berada disekitar garis konvergensi training yang berada di antara nilai 0.7 dan 0.8.

% Grafik sepuluh prediksi
\par Parameter yang diukur memiliki karakteristik prediksi sebagai berikut :
\begin{enumerate}
    \item Proto
    \begin{figure}%[H]
        \centering
        \includegraphics[width=0.7\textwidth]{public/assets/img/lstmm_neris_pred1.png}
        \caption{Grafik Prediksi Proto neris}
        \label{fig:lstmm_neris_pred1}
    \end{figure}
    
    \par Header Proto merupakan salah satu bagian dari paket jaringan yang mendefinisikan jenis protokol jaringan yang digunakan. Dapat dilihat dari grafik prediksi mencakup sebagaian besar nilai tertinggi diantara 0.8dan 1.0, sedangkan beberapa titik dibawah tidak tedeteksi.
    
    \item SrcAddr
    \begin{figure}%[H]
        \centering
        \includegraphics[width=0.7\textwidth]{public/assets/img/lstmm_neris_pred2.png}
        \caption{Grafik Prediksi SrcAddr neris}
        \label{fig:lstmm_neris_pred2}
    \end{figure}
    
    \par Header SrcAddr merupakan bagian IP header dari paket jaringan yang mendefinisikan alamat IP sumber. Dapat dilihat dari grafik bahwa sebagian besar hasil prediksi hanya mencakup nilai rendah disekitar 0.1 saja, sedangkan trafik asli tersebar merata dengan interval 0.1.
    
    \item DstAddr
    \begin{figure}%[H]
        \centering
        \includegraphics[width=0.7\textwidth]{public/assets/img/lstmm_neris_pred3.png}
        \caption{Grafik Prediksi DstAddr neris}
        \label{fig:lstmm_neris_pred3}
    \end{figure}
    
    \par Header DstAddr ini merupakan bagian IP header dari paket jaringan yang mendefinsikan alamat IP tujuan. Dapat dilihat dari grafik bahwa sebagian besar hasil prediksi dapat mencakup keseluruhan trafik asli dan dibagian antara 1.0 dan 0.8 tidak terdeteksi.
    
    \item Sport
    \begin{figure}%[H]
        \centering
        \includegraphics[width=0.7\textwidth]{public/assets/img/lstmm_neris_pred4.png}
        \caption{Grafik Prediksi Sport neris}
        \label{fig:lstmm_neris_pred4}
    \end{figure}
    
    \par Header Sport ini merupakan bagian dari IP header yang mendefinisikan port sumber. Dapat dilihat dari grafik hasil pendeteksian port sumber mencakup keseluruhan, walaupun ada beberapa bagian yang tidak dicakup terutama dibagian nilai tertinggi yakni disekitar 0.0 dan sekitar 1.0.
    
    \item Dport
    \begin{figure}%[H]
        \centering
        \includegraphics[width=0.7\textwidth]{public/assets/img/lstmm_neris_pred5.png}
        \caption{Grafik Prediksi Dport neris}
        \label{fig:lstmm_neris_pred5}
    \end{figure}
    
    \par Header Dport ini merupakan bagian dari IP header yang mendefinisikan port tujuan. Dapat dilihat dari grafik prediksi bahwa cakupan data prediksi mencakup keseluruhan sedangkan beberapa bagian yakni sekitar 1.0 dan 0.8 sebagian besar tidak terdeteksi
    
    \item Dir
    \begin{figure}%[H]
        \centering
        \includegraphics[width=0.7\textwidth]{public/assets/img/lstmm_neris_pred6.png}
        \caption{Grafik Prediksi Dir neris}
        \label{fig:lstmm_neris_pred6}
    \end{figure}
    
    \par Header Dir merupakan arah yang menunjukan arah transmisi dari paket baik itu multicast, maupun unicast. Dapat dilihat dari grafik hasil prediksi bahwa keseluruhan trafik dicakup kecuali disekitar kelas 0.8 dan 1.0.
    
    \item TotPkts
    \begin{figure}%[H]
        \centering
        \includegraphics[width=0.7\textwidth]{public/assets/img/lstmm_neris_pred7.png}
        \caption{Grafik Prediksi TotPkts neris}
        \label{fig:lstmm_neris_pred7}
    \end{figure}
    
    \par Header TotPkts merupakan bagian dari paket header yang menjelaskan jumlah paket dalam suatu ethernet frame yang dikirim per paket. Dari grafik hasil prediksi dapat diamati bahwa keseluruhan variasi totpkts teredeteksi sedangkan diukuran sekitar 0.8 dan 1.0 tidak terdeteksi.
    
    \item TotBytes
    \begin{figure}%[H]
        \centering
        \includegraphics[width=0.7\textwidth]{public/assets/img/lstmm_neris_pred8.png}
        \caption{Grafik Prediksi TotBytes neris}
        \label{fig:lstmm_neris_pred8}
    \end{figure}
    
    \par Header TotBytes merupakan bagian dari paket header yang menjelaskan jumlah bytes dalam satu ethernet frame yang dikirim per paket. Dapat dilihat dari grafik hasil prediksi diatas bahwa sebagian besar prediksi mencakup semua kecuali dibagian 0.8 dan 1.0.
    
    \item SrcBytes
    \begin{figure}%[H]
        \centering
        \includegraphics[width=0.7\textwidth]{public/assets/img/lstmm_neris_pred9.png}
        \caption{Grafik Prediksi SrcBytes neris}
        \label{fig:lstmm_neris_pred9}
    \end{figure}
    
    \par Header SrcBytes merupakan bagian dari paket header yang menjelaskan ukuran byte source. Dapat diamati dari grafik diatas bahwa keseluruhan hasil prediksi mencakup semua trafik kecuali dibagian terendah.
    \item Label
    \begin{figure}%[H]
        \centering
        \includegraphics[width=0.7\textwidth]{public/assets/img/lstmm_neris_pred10.png}
        \caption{Grafik Prediksi Label neris}
        \label{fig:lstmm_neris_pred10}
    \end{figure}
    
    \par Pada bagian ini berisi label dari dataset yang sudah di angka kan. Dapat dilhat dari grafik diatas bahwa ukuran yang terdeteksi kecuali dibagian trafik tertinggi diantara 0.9 dan 1.0 dan diantara 0.6.
\end{enumerate}

\par Berdasarakan grafik keseluruhan dari hasil prediksi kesepuluh parameter \textit{LSTM Multivariate Prediction}, dapat dilihat bahwa pada beberapa parameter prediksi terdapat beberapa parmaeter yang berhasil di prediksi dan beberapa parameter pula yang tidak dapat terdeteksi.

\par Berdasarakan grafik diatas keseluruhan dari hasil prediksi kesepuluh parameter \textit{LSTM Multivariate Prediction}, dapat dilihat bahwa pada beberapa parameter prediksi terdapat beberapa parmaeter yang berhasil di prediksi dan beberapa parameter pula yang tidak dapat terdeteksi.

% Tabel Hasil
\begin{table}%[H]
\centering
\caption{Tabel Hasil LSTMM Neris}
\begin{tabelkeras}
\hline
0  &  0.065239 &  0.654737 &                 0.065239 &                  0.654737 &  0.119845 &  0.449928 &             0.119845 &              0.449928 \\
5  &  0.034839 &  0.733986 &                 0.034839 &                  0.733986 &  0.046002 &  0.741092 &             0.046002 &              0.741092 \\
10 &  0.031048 &  0.442449 &                 0.031048 &                  0.442449 &  0.040226 &  0.743360 &             0.040226 &              0.743360 \\
15 &  0.029231 &  0.388196 &                 0.029231 &                  0.388196 &  0.037856 &  0.741544 &             0.037856 &              0.741544 \\
20 &  0.028663 &  0.376175 &                 0.028663 &                  0.376175 &  0.036085 &  0.739152 &             0.036085 &              0.739152 \\
25 &  0.028420 &  0.373686 &                 0.028420 &                  0.373686 &  0.034885 &  0.740980 &             0.034885 &              0.740980 \\
30 &  0.027671 &  0.381238 &                 0.027671 &                  0.381238 &  0.034157 &  0.741424 &             0.034157 &              0.741424 \\
35 &  0.028124 &  0.385267 &                 0.028124 &                  0.385267 &  0.033609 &  0.739756 &             0.033609 &              0.739756 \\
40 &  0.028225 &  0.383293 &                 0.028225 &                  0.383293 &  0.033262 &  0.738816 &             0.033262 &              0.738816 \\
45 &  0.028852 &  0.377458 &                 0.028852 &                  0.377458 &  0.032910 &  0.739264 &             0.032910 &              0.739264 \\
\hline
\end{tabelkeras}
\label{table:lstmm_neris}
\end{table}

\par Berdasarkan tabel \ref{table:lstmm_neris} diatas dapat diamati bahwa penurunan Loss terjadi dan peningkatan terjadi. Pada sisi akurasi dapat dilihat bahwa akurasi tertinggi yang dicapai model adalah 0.75. Hal ini dikarenakan beberapa parameter tidak mempengaruhi satu sama lain, namun berdasarkan keseluruhan hasil prediksi yang diperoleh dapat digunakan untuk melakukan profiling.

\subsubsection{Hasil Training Rbot Botnet dengan Multivariate Prediction LSTM}

\begin{figure}%[H]
    \centering
    \includegraphics[width=0.7\textwidth]{public/assets/img/lstmm_rbot_loss.png}
    \caption{Loss model LSTM Rbot Multivariate Prediction}
    \label{fig:lstmm_rbot_loss}
\end{figure}

\begin{figure}%[H]
    \centering
    \includegraphics[width=0.7\textwidth]{public/assets/img/lstmm_rbot_acc.png}
    \caption{Akurasi model LSTM Rbot Multivariate Prediction}
    \label{fig:lstmm_rbot_acc}
\end{figure}

\par Berdasarkan dari grafik diatas dapat diamati bahwa hasil training konvergen diantara 0.04 pada grafik testing dan 0.05 pada grafik training.
\par Dapat dilihat dari grafik loss mengalami pemusatan pada satu sumbu y baik pada grafik \textit{Loss} maupun pada grafik \textit{Accuracy}. Pada grafik akurasi terjadi fluktuasi pada hasil testing mulai pada epoch ke 25 sampai ke 50.

% Grafik sepuluh prediksi
% Grafik sepuluh prediksi
\par Parameter yang diukur memiliki karakteristik prediksi sebagai berikut :
\begin{enumerate}
    \item Proto
    \penalty 10000
    \begin{figure}%[H]
        \centering
        \includegraphics[width=0.7\textwidth]{public/assets/img/lstmm_rbot_pred1.png}
        \caption{Grafik Prediksi Proto rbot}
        \label{fig:lstmm_rbot_pred1}
    \end{figure}
    
    \par Bagian ini merupakan salah satu bagian dari paket jaringan yang mendefinisikan jenis protokol jaringan yang digunakan. Dari grafik diatas dapat dilihat cakupan dari prediksi ada disekitar 0.8 sampai 0.9 sedangkan diluar dari itu tidak terdeteksi.
    
    \item SrcAddr \nobreak
    \begin{figure}%[H]
        \centering
        \includegraphics[width=0.7\textwidth]{public/assets/img/lstmm_rbot_pred2.png}
        \caption{Grafik Prediksi SrcAddr rbot}
        \label{fig:lstmm_rbot_pred2}
    \end{figure}
    
    \par Bagian ini merupakan bagian IP header dari paket jaringan yang mendefinisikan alamat IP sumber. Dapat dilihat dari grafik diatas hanya satu bagian yang terdeteksi yakni 0.1, sedangkan sumber lain tidak terdeteksi.
    
    \item DstAddr
    \begin{figure}%[H]
        \centering
        \includegraphics[width=0.7\textwidth]{public/assets/img/lstmm_rbot_pred3.png}
        \caption{Grafik Prediksi DstAddr rbot}
        \label{fig:lstmm_rbot_pred3}
    \end{figure}
    
    \par Bagian ini merupakan bagian IP header dari paket jaringan yang mendefinsikan alamat IP tujuan. Dapat dilihat dari grafik diatas bahwa sebagian besar trafik terdeteksi kecuali dibagian antara 0.8 dan 1.0.
    
    \item Sport
    \begin{figure}%[H]
        \centering
        \includegraphics[width=0.7\textwidth]{public/assets/img/lstmm_rbot_pred4.png}
        \caption{Grafik Prediksi Sport rbot}
        \label{fig:lstmm_rbot_pred4}
    \end{figure}
    
    \par Bagian ini merupakan bagian dari IP header yang mendefinisikan port sumber. Dapat dilihat dari grafik diatas bahwa sebagian besar trafik terdeteksi kecuali di antara 0.0 dan 1.0.
    
    \item Dport
    \begin{figure}%[H]
        \centering
        \includegraphics[width=0.7\textwidth]{public/assets/img/lstmm_rbot_pred5.png}
        \caption{Grafik Prediksi Dport rbot}
        \label{fig:lstmm_rbot_pred5}
    \end{figure}
    
    \par Bagian ini merupakan bagian dari IP header yang mendefinisikan port tujuan. Dapat dilihat dari grafik prediksi diatas hasil perdiksi dapat mencakup keseluruhan trafik kecuali bagian tertinggi 1.0.
    
    \item Dir
    \begin{figure}%[H]
        \centering
        \includegraphics[width=0.7\textwidth]{public/assets/img/lstmm_rbot_pred6.png}
        \caption{Grafik Prediksi Dir rbot}
        \label{fig:lstmm_rbot_pred6}
    \end{figure}
    
    \par Bagian ini merupakan arah yang menunjukan arah transmisi dari paket baik itu multicast, maupun unicast. Dapat dilihat dari grafik prediksi diatas keseluruhan hasil prediksi dapat mencakup trafik kecuali 0.8 dan 1.0.
    
    \item TotPkts
    \begin{figure}%[H]
        \centering
        \includegraphics[width=0.7\textwidth]{public/assets/img/lstmm_rbot_pred7.png}
        \caption{Grafik Prediksi TotPkts rbot}
        \label{fig:lstmm_rbot_pred7}
    \end{figure}
    
    \par Bagian ini merupakan bagian dari paket header yang menjelaskan jumlah paket dalam suatu ethernet frame yang dikirim per paket. Dari grafik hasil prediksi dapat dilihat seluruh trafik dapat diprediksi kecuali diantar 0.8 dan 1.0
    
    \item TotBytes
    \begin{figure}%[H]
        \centering
        \includegraphics[width=0.7\textwidth]{public/assets/img/lstmm_rbot_pred8.png}
        \caption{Grafik Prediksi TotBytes rbot}
        \label{fig:lstmm_rbot_pred8}
    \end{figure}
    
    \par Bagian ini merupakan bagian dari paket header yang menjelaskan jumlah bytes dalam satu ethernet frame yang dikirim per paket. Dapat dilihat dari grafik prediksi diatas bahwa secara keseluruhan dapat diprediksi kecuali nilai diantara 0.8 dan 1.0.
    
    \item SrcBytes
    \begin{figure}%[H]
        \centering
        \includegraphics[width=0.7\textwidth]{public/assets/img/lstmm_rbot_pred9.png}
        \caption{Grafik Prediksi SrcBytes rbot}
        \label{fig:lstmm_rbot_pred9}
    \end{figure}
    
    \par Bagian ini merupakan bagian dari paket header yang menjelaskan ukuran byte source. Dapat dilihat dari grafik diatas data SrcBytes dapat diprediksi kecuali nilai terendah diantara 0.0.
    \item Label
    \begin{figure}%[H]
        \centering
        \includegraphics[width=0.7\textwidth]{public/assets/img/lstmm_rbot_pred10.png}
        \caption{Grafik Prediksi Label rbot}
        \label{fig:lstmm_rbot_pred10}
    \end{figure}
    
    \par Bagian ini merupakan label dari dataset yang sudah di angka kan. Dapat dilihat dari grafik hasil prediksi diatas, bahwa secara keseluruhan trafik dapat diprediksi kecuali trafik diantara 1.0 dan 0.6.
\end{enumerate}

\par Dapat diamati grafik keseluruhan dari hasil prediksi kesepuluh parameter \textit{LSTM Multivariate Prediction}, dapat dilihat bahwa pada beberapa parameter prediksi terdapat beberapa parmaeter yang berhasil di prediksi dan beberapa parameter pula yang tidak dapat terdeteksi.

% Tabel Hasil
\begin{table}%[H]
\centering
\caption{Tabel Hasil LSTMM Rbot}
\begin{tabelkeras}
\hline
0  &  0.096960 &  0.717496 &                 0.096960 &                  0.717496 &  0.124205 &  0.469576 &             0.124205 &              0.469576 \\
5  &  0.039409 &  0.815303 &                 0.039409 &                  0.815303 &  0.046257 &  0.731320 &             0.046257 &              0.731320 \\
10 &  0.037089 &  0.813762 &                 0.037089 &                  0.813762 &  0.043225 &  0.733896 &             0.043225 &              0.733896 \\
15 &  0.036373 &  0.778298 &                 0.036373 &                  0.778298 &  0.041665 &  0.736736 &             0.041665 &              0.736736 \\
20 &  0.036282 &  0.803957 &                 0.036282 &                  0.803957 &  0.040376 &  0.743860 &             0.040376 &              0.743860 \\
25 &  0.035489 &  0.702191 &                 0.035489 &                  0.702191 &  0.039543 &  0.745740 &             0.039543 &              0.745740 \\
30 &  0.034391 &  0.765884 &                 0.034391 &                  0.765884 &  0.038935 &  0.744652 &             0.038935 &              0.744652 \\
35 &  0.034246 &  0.698371 &                 0.034246 &                  0.698371 &  0.038311 &  0.747672 &             0.038311 &              0.747672 \\
40 &  0.033922 &  0.736452 &                 0.033922 &                  0.736452 &  0.038004 &  0.747848 &             0.038004 &              0.747848 \\
45 &  0.033341 &  0.700033 &                 0.033341 &                  0.700033 &  0.037738 &  0.747188 &             0.037738 &              0.747188 \\
\hline
\end{tabelkeras}
\label{table:lstmm_rbot}
\end{table}

\par Tabel \ref{table:lstmm_rbot} diatas menunjukkan bahwa penurunan Loss terjadi dan peningkatan terjadi. Pada sisi akurasi dapat dilihat bahwa akurasi tertinggi yang dicapai model adalah 0.75. Hal ini dikarenakan beberapa parameter tidak mempengaruhi satu sama lain, namun berdasarkan keseluruhan hasil prediksi yang diperoleh dapat digunakan untuk melakukan profiling.

\subsubsection{Hasil Training LSTM Svchosta 4 Directional Header}
% Grafik model

\begin{figure}%[H]
    \centering
    \includegraphics[width=0.7\textwidth]{public/assets/img/lstm4_svchosta_loss.png}
    \caption{Loss model LSTM Svchosta 4 Directional Header}
    \label{fig:lstm4_svchosta_loss}
\end{figure}
\begin{figure}%[H]
    \centering
    \includegraphics[width=0.7\textwidth]{public/assets/img/lstm4_svchosta_acc.png}
    \caption{Akurasi model LSTM Svchosta 4 Directional Header}
    \label{fig:lstm4_svchosta_acc}
\end{figure}

\par Grafik training diatas menunjukkan bahwa pada sisi loss terjadi konvergensi baik pada sisi \textit{Loss} maupun pada sisi \textit{Akurasi}. Semakin banyak \textit{epoch} yang terjadi maka semakin kecil \textit{Loss} yang diperoleh. Semakin banyak \textit{epoch} yang terjadi maka semakin tinggi \textit{Accuracy} yang diperoleh.

% Grafik prediksi IP
\begin{figure}%[H]
    \centering
    \buatsubgrafik{public/assets/img/lstm4_svchosta_pred1.png}{Prediksi IPSrc model LSTM4 svchosta}{.45}{fig:lstm4_svchosta_pred1a}
    \buatsubgrafik{public/assets/img/lstm4_svchosta_pred2.png}{Prediksi IPDst model LSTM4 svchosta}{.45}{fig:lstm4_svchosta_pred1b}
    \caption{Prediksi IP Address model LSTM4 svchosta}
    \label{fig:lstm4_svchosta_pred1}
\end{figure}

\par Grafik hasil prediksi \ref{fig:lstm4_svchosta_pred1} diperoleh dengan mengambil 500 sampel dari dataset, dapat diamati bahwa hampir secara keseluruhan hasil prediksi cocok dengan data yang di inputkan. Grafik prediksi \textbf{IPSrc} memiliki divergensi yang tinggi berarti banyak IP yang mengakses Server. Sedangkan Grafik IPDst yang lurus menandakan hanya ada satu alamat IP yang dituju yakni alamat IP Server.

% Grafik prediksi Port

\begin{figure}%[H]
    \centering
    \buatsubgrafik{public/assets/img/lstm4_svchosta_pred3.png}{Prediksi PortSrc model LSTM4 svchosta}{.45}{fig:lstm4_svchosta_pred2a}
    \buatsubgrafik{public/assets/img/lstm4_svchosta_pred4.png}{Prediksi PortDst model LSTM4 svchosta}{.45}{fig:lstm4_svchosta_pred2b}
    \caption{Prediksi Port model LSTM4 svchosta}
    \label{fig:lstm4_svchosta_pred2}
\end{figure}

\par Pada grafik hasil prediksi \ref{fig:lstm4_svchosta_pred2} dapat dilihat bahwa secara keseluruhan hasil prediksi cocok dengan data yang di inputkan. Grafik prediksi \textbf{PortSrc} memiliki divergensi yang tinggi berarti banyak Port sumber yang mengakses server. Sedangkan Grafik PortDst yang lurus menandakan hanya ada satu Port yang dituju yakni Port HTTP.

% Tabel Hasil
\begin{table}%[H]
\centering
\caption{Tabel Hasil LSTM4 Svchosta}
\begin{tabelkeras}
\hline
0   &  0.432725 &    0.000 &                 0.432725 &                     0.000 &  0.445629 &  0.000 &             0.445629 &                 0.000 \\
100 &  0.038453 &    0.646 &                 0.038453 &                     0.646 &  0.039711 &  0.650 &             0.039711 &                 0.650 \\
200 &  0.037909 &    0.982 &                 0.037909 &                     0.982 &  0.038660 &  0.814 &             0.038660 &                 0.814 \\
300 &  0.036691 &    0.670 &                 0.036691 &                     0.670 &  0.037372 &  0.772 &             0.037372 &                 0.772 \\
400 &  0.032788 &    0.670 &                 0.032788 &                     0.670 &  0.033537 &  0.778 &             0.033537 &                 0.778 \\
500 &  0.025009 &    0.672 &                 0.025009 &                     0.672 &  0.025460 &  0.810 &             0.025460 &                 0.810 \\
600 &  0.012822 &    0.982 &                 0.012822 &                     0.982 &  0.014406 &  0.856 &             0.014406 &                 0.856 \\
700 &  0.004456 &    0.982 &                 0.004456 &                     0.982 &  0.007659 &  0.846 &             0.007659 &                 0.846 \\
800 &  0.002419 &    0.672 &                 0.002419 &                     0.672 &  0.006625 &  0.868 &             0.006625 &                 0.868 \\
900 &  0.002374 &    0.982 &                 0.002374 &                     0.982 &  0.006360 &  0.848 &             0.006360 &                 0.848 \\
\hline
\end{tabelkeras}
\label{table:lstm4_svchosta}
\end{table}

\par Berdasarkan tabel \ref{table:lstm4_svchosta} diatas dapat diamati bahwa seiring dengan meningkatnya jumlah epoch Nilai \textit{Loss} baik \textit{Value Loss} maupun \textit{Loss} menurun. Begitu pula dengan meningkatnya jumlah epoch Nilai \textit{Accuracy} baik \textit{Value Accuracy} maupun \textit{Accuracy}. Dapat diamati pada tabel diatas akurasi tertinggi yang dicapai adalah 98.2\% pada sisi testing dan 84.8\%.

\subsubsection{Hasil Training LSTM Neris 4 Directional Header}
% Grafik model
\begin{figure}
    \centering
    \includegraphics[width=0.7\textwidth]{public/assets/img/lstm4_neris_loss.png}
    \caption{Loss model LSTM Neris 4 Directional Header}
    \label{fig:lstm4_neris_loss}
\end{figure}
\begin{figure}%[H]
    \centering
    \includegraphics[width=0.7\textwidth]{public/assets/img/lstm4_neris_acc.png}
    \caption{Akuasi model LSTM Neris 4 Directional Header}
    \label{fig:lstm4_neris}
\end{figure}

\par Grafik hasil training \ref{fig:lstm4_neris} diatas menunjukkan bahwa pada sisi loss terjadi konvergensi baik pada sisi \textit{Loss} maupun pada sisi \textit{Akurasi}. Semakin banyak \textit{epoch} yang terjadi maka semakin kecil \textit{Loss} yang diperoleh. Semakin banyak \textit{epoch} yang terjadi maka semakin tinggi \textit{Accuracy} yang diperoleh.

% Grafik prediksi IP
\begin{figure}
    \centering
    \buatsubgrafik{public/assets/img/lstm4_neris_pred1.png}{Prediksi IPSrc model LSTM4 Neris}{.4}{fig:lstm4_neris_pred1a}
    \buatsubgrafik{public/assets/img/lstm4_neris_pred2.png}{Prediksi IPDst model LSTM4 Neris}{.4}{fig:lstm4_neris_pred1b}
    \caption{Prediksi IP Address model LSTM4 Neris}
    \label{fig:lstm4_neris_pred1}
\end{figure}

\par Pada grafik hasil prediksi \ref{fig:lstm4_neris_pred1} dapat diamati bahwa hampir secara keseluruhan hasil prediksi cocok dengan data yang di inputkan. Grafik prediksi \textbf{IPSrc} memiliki divergensi yang tinggi berarti banyak IP yang mengakses Server. Sedangkan Grafik IPDst yang lurus menandakan hanya ada satu alamat IP yang dituju yakni alamat IP Server.

% Grafik prediksi Port
\begin{figure}
    \centering
    \buatsubgrafik{public/assets/img/lstm4_neris_pred3.png}{Prediksi PortSrc model LSTM4 Neris}{.4}{fig:lstm4_neris_pred2a}
    \buatsubgrafik{public/assets/img/lstm4_neris_pred3.png}{Prediksi PortSrc model LSTM4 Neris}{.4}{fig:lstm4_neris_pred2b}
    \caption{Prediksi Port model LSTM4 Neris}
    \label{fig:lstm4_neris_pred2}
\end{figure}


\par Berdasarkan grafik hasil prediksi \ref{fig:lstm4_neris_pred2} dengan mengambil 500 sampel dari dataset dapat diamati bahwa hampir secara keseluruhan hasil prediksi cocok dengan data yang di inputkan. Grafik prediksi \textbf{PortSrc} memiliki divergensi yang tinggi berarti banyak Port sumber yang mengakses server. Sedangkan Grafik PortDst yang lurus menandakan hanya ada satu Port yang dituju yakni Port HTTP.

% Tabel Hasil
\begin{table}%[H]
\centering
\caption{Tabel Hasil LSTM4 Neris}
\begin{tabelkeras}
\hline
0   &  0.399342 &    0.000 &                 0.399342 &                      2.464162 &  0.405103 &  0.120 &             0.405103 &                  2.539193 \\
100 &  0.049418 &    0.610 &                 0.049418 &                      2.268167 &  0.050036 &  0.642 &             0.050036 &                  2.260185 \\
200 &  0.048140 &    0.826 &                 0.048140 &                      2.266545 &  0.048374 &  0.882 &             0.048374 &                  2.258518 \\
300 &  0.042422 &    0.786 &                 0.042422 &                      2.258401 &  0.041996 &  0.856 &             0.041996 &                  2.249844 \\
400 &  0.026611 &    0.806 &                 0.026611 &                      2.243014 &  0.024903 &  0.898 &             0.024903 &                  2.234774 \\
500 &  0.018720 &    0.980 &                 0.018720 &                      2.237860 &  0.018771 &  0.936 &             0.018771 &                  2.230748 \\
600 &  0.012459 &    0.984 &                 0.012459 &                      2.232215 &  0.013559 &  0.938 &             0.013559 &                  2.226480 \\
700 &  0.006294 &    0.990 &                 0.006294 &                      2.227622 &  0.008677 &  0.936 &             0.008677 &                  2.223640 \\
800 &  0.004994 &    0.990 &                 0.004994 &                      2.227298 &  0.008039 &  0.934 &             0.008039 &                  2.223908 \\
900 &  0.004327 &    0.830 &                 0.004327 &                      2.227232 &  0.007939 &  0.930 &             0.007939 &                  2.223906 \\
\hline
\end{tabelkeras}
\label{table:lstm4_neris}
\end{table}

\par Berdasarkan tabel \ref{table:lstm4_neris} diatas dapat diamati bahwa seiring dengan meningkatnya jumlah epoch Nilai \textit{Loss} baik \textit{Value Loss} maupun \textit{Loss} menurun. Begitu pula dengan meningkatnya jumlah epoch Nilai \textit{Accuracy} baik \textit{Value Accuracy} maupun \textit{Accuracy}. Dapat diamati pada tabel diatas akurasi tertinggi yang dicapai adalah 94.0\% pada sisi testing dan 83.0\%.

\subsubsection{Hasil Training LSTM Rbot 4 Directional Header}
\begin{figure}%[H]
    \centering
    \includegraphics[width=0.7\textwidth]{public/assets/img/lstm4_rbot_loss.png}
    \caption{Loss model LSTM Rbot 4 Directional Header}
    \label{fig:lstm4_rbot_loss}
\end{figure}

\begin{figure}%[H]
    \centering
    \includegraphics[width=0.7\textwidth]{public/assets/img/lstm4_rbot_acc.png}
    \caption{Akurasi model LSTM Rbot 4 Directional Header}
    \label{fig:lstm4_rbot_acc}
\end{figure}

\par Mengacu pada grafik hasil training diatas dapat dilihat bahwa pada sisi loss terjadi konvergensi baik pada sisi \textit{Loss} maupun pada sisi \textit{Akurasi}. Semakin banyak \textit{epoch} yang terjadi maka semakin kecil \textit{Loss} yang diperoleh. Semakin banyak \textit{epoch} yang terjadi maka semakin tinggi \textit{Accuracy} yang diperoleh.

% Grafik prediksi IP
\begin{figure}%[H]
    \centering
    \buatsubgrafik{public/assets/img/lstm4_rbot_pred1.png}{Prediksi IPSrc model LSTM4 rbot}{.45}{fig:lstm4_rbot_pred1a}
    \buatsubgrafik{public/assets/img/lstm4_rbot_pred2.png}{Prediksi IPDst model LSTM4 rbot}{.45}{fig:lstm4_rbot_pred1b}
    \caption{Prediksi IP Address model LSTM 4 rbot}
    \label{fig:lstm4_rbot_pred1}
\end{figure}

\par Grafik hasil prediksi \ref{fig:lstm4_rbot_pred1} menunjukkan bahwa hampir secara keseluruhan hasil prediksi cocok dengan data yang di inputkan. Grafik prediksi \textbf{IPSrc} memiliki divergensi yang tinggi berarti banyak IP yang mengakses Server. Sedangkan Grafik IPDst yang lurus menandakan hanya ada satu alamat IP yang dituju yakni alamat IP Server.

% Grafik prediksi Port
\begin{figure}%[H]
    \centering
    \buatsubgrafik{public/assets/img/lstm4_rbot_pred3.png}{Prediksi PortSrc model LSTM4 Rbot}{.45}{fig:lstm4_rbot_pred2a}
    \buatsubgrafik{public/assets/img/lstm4_rbot_pred4.png}{Prediksi PortDst model LSTM4 Rbot}{.45}{fig:lstm4_rbot_pred2b}
    \caption{Prediksi Port model LSTM4 Rbot}
    \label{fig:lstm4_rbot_pred2}
\end{figure}

\par Grafik \ref{fig:lstm4_rbot_pred2} diperoleh dengan mengambil 500 sampel dari dataset dapat diamati bahwa hampir secara keseluruhan hasil prediksi cocok dengan data yang di inputkan. Grafik prediksi \textbf{PortSrc} memiliki divergensi yang tinggi berarti banyak Port sumber yang mengakses server. Sedangkan Grafik PortDst yang lurus menandakan hanya ada satu Port yang dituju yakni Port HTTP.


% Tabel Hasil
\begin{table}%[H]
\centering
\caption{Tabel Hasil LSTM4 Rbot}
\begin{tabelkeras}
\hline
0   &  0.393062 &    0.812 &                 0.393062 &                      2.264847 &  0.395210 &  0.582 &             0.395210 &                  4.385572 \\
100 &  0.019342 &    0.980 &                 0.019342 &                      2.139255 &  0.041992 &  0.684 &             0.041992 &                  2.115251 \\
200 &  0.019122 &    0.996 &                 0.019122 &                      2.138949 &  0.041096 &  0.838 &             0.041096 &                  2.114622 \\
300 &  0.018246 &    0.994 &                 0.018246 &                      2.137809 &  0.039282 &  0.856 &             0.039282 &                  2.112313 \\
400 &  0.015216 &    0.994 &                 0.015216 &                      2.134190 &  0.032752 &  0.832 &             0.032752 &                  2.105142 \\
500 &  0.007832 &    0.998 &                 0.007832 &                      2.129096 &  0.019392 &  0.816 &             0.019392 &                  2.096398 \\
600 &  0.005506 &    0.330 &                 0.005506 &                      2.128211 &  0.016743 &  0.806 &             0.016743 &                  2.094837 \\
700 &  0.004807 &    0.328 &                 0.004807 &                      2.127276 &  0.014481 &  0.810 &             0.014481 &                  2.092351 \\
800 &  0.003952 &    0.328 &                 0.003952 &                      2.126126 &  0.010795 &  0.812 &             0.010795 &                  2.088924 \\
900 &  0.002218 &    0.330 &                 0.002218 &                      2.125234 &  0.007115 &  0.856 &             0.007115 &                  2.086706 \\
\hline
\end{tabelkeras}
\label{table:lstm4_rbot}
\end{table}

\par Pada tabel \ref{table:lstm4_rbot} diatas dapat diamati bahwa seiring dengan meningkatnya jumlah epoch Nilai \textit{Loss} baik \textit{Value Loss} maupun \textit{Loss} menurun. Begitu pula dengan meningkatnya jumlah epoch Nilai \textit{Accuracy} baik \textit{Value Accuracy} maupun \textit{Accuracy}. Dapat diamati pada tabel diatas akurasi tertinggi yang dicapai adalah 94.0\% pada sisi testing dan 83.0\%.

\subsection{Perbandingan Metode LSTM}
\par Berdasarkan dari ketiga metode LSTM yang digunakan, dapat kita amati beberapa karakteristik perbedaan sebagai berikut :
\begin{table}%[H]
\begin{tabularx}{\textwidth}{
                |c
                |p{0.68\textwidth}|}
\hline
Metode & Karakteristik \\
\hline
Sentiment Analysis & Tidak konvergen, sehingga tidak perlu dilanjutkan \\
\hline
Multivariate Analysis & Konvergen, tetapi banyak parameter yang mengganggu proses training \\
\hline
4 Directional Header & Konvergen dan karena hanya menggunakan bagian yang benar-benar diperhitungkan sehingga hasil yang diperoleh akurat \\
\hline
\end{tabularx}
\caption{Data hasil perbandingan metode LSTM}
\label{table:data_perbandingan_lstm}
\end{table}
\subsection{Hasil Training Data CNN}
\par Pada jaringan CNN terjadi proses filter packet payload. Disini ditentukan apakah payload mengandung virus atau tidak. Proses training dilakukan dengan mengubah-ubah parameter learning\_rate. Hal ini dilakukan untuk mengamati perubahan konvergensi proses training dan testing dari CNN.

% Grafik
\newcolumntype{Y}{>{\centering\arraybackslash}X}
% BEGIN TABEL
\subsubsection{Hasil Training CNN Svchosta}
\begin{table}%[H]
\centering
\caption{Tabel Hasil Training CNN Svchosta}
\begin{tabularx}{\textwidth}{|*{9}{Y|}}
\hline
    \multirow{2}{*}{Epoch} 
  & \multicolumn{4}{c|}{Akurasi}
  & \multicolumn{4}{c|}{Loss} \\
\cline{2-9}
   &      1e-1 &      1e-2 &      1e-3 &      1e-4 &      1e-1 &      1e-2 &      1e-3 &      1e-4 \\
\cline{1-9}
0  & 0.800 & 0.798 & 0.794 & 0.206 & 0.515 & 0.537 & 0.604 & 0.893 \\
5  & 1.000 & 0.998 & 0.889 & 0.229 & 0.007 & 0.057 & 0.453 & 0.816 \\
10 & 1.000 & 1.000 & 0.950 & 0.308 & 0.003 & 0.027 & 0.337 & 0.750 \\
15 & 1.000 & 1.000 & 0.973 & 0.438 & 0.002 & 0.018 & 0.255 & 0.691 \\
20 & 1.000 & 1.000 & 0.980 & 0.587 & 0.001 & 0.013 & 0.198 & 0.645 \\
25 & 1.000 & 1.000 & 0.984 & 0.741 & 0.001 & 0.011 & 0.160 & 0.596 \\
30 & 1.000 & 1.000 & 0.991 & 0.873 & 0.001 & 0.009 & 0.131 & 0.556 \\
35 & 1.000 & 1.000 & 0.993 & 0.941 & 0.001 & 0.008 & 0.110 & 0.518 \\
40 & 1.000 & 1.000 & 0.995 & 0.973 & 0.001 & 0.007 & 0.096 & 0.485 \\
45 & 1.000 & 1.000 & 0.995 & 0.986 & 0.001 & 0.006 & 0.083 & 0.455 \\
\hline
\end{tabularx}
\label{table:cnn_svchosta_train}
\end{table}
\par Dapat dilihat dari tabel hasil train \ref{table:cnn_svchosta_train}  diatas, semakin banyak epoch maka semakin tinggi akurasi dan semakin rendah loss nya. Lama akurasi mencapai titik tertinggi, dan loss mencapai titik terendah ditentukan oleh \textit{learning rate} .

\par Dapat diamati berdasarkan parameter \textit{learning rate}, dimana semakin besar nilai \textit{learning rate} maka semakin cepat pula akurasi mencapai titik tertinggi, dan loss mencapai titik terendah.
\begin{table}%[H]
\centering
\caption{Tabel Hasil Testing CNN Svchosta}
\begin{tabularx}{\textwidth}{|*{9}{Y|}}
\hline
    \multirow{2}{*}{Epoch} 
  & \multicolumn{4}{c|}{Akurasi}
  & \multicolumn{4}{c|}{Loss} \\
\cline{2-9}
   &      1e-1 &      1e-2 &      1e-3 &      1e-4 &      1e-1 &      1e-2 &      1e-3 &      1e-4 \\
\cline{1-9}
0  & 0.980 & 0.953 & 0.861 & 0.141 & 0.234 & 0.267 & 0.551 & 0.917 \\
5  & 1.000 & 1.000 & 0.920 & 0.169 & 0.008 & 0.050 & 0.417 & 0.832 \\
10 & 1.000 & 1.000 & 0.966 & 0.277 & 0.003 & 0.024 & 0.306 & 0.762 \\
15 & 1.000 & 1.000 & 0.981 & 0.407 & 0.002 & 0.016 & 0.228 & 0.701 \\
20 & 1.000 & 1.000 & 0.987 & 0.601 & 0.001 & 0.013 & 0.176 & 0.649 \\
25 & 1.000 & 1.000 & 0.991 & 0.782 & 0.001 & 0.009 & 0.140 & 0.603 \\
30 & 1.000 & 1.000 & 0.995 & 0.900 & 0.001 & 0.008 & 0.115 & 0.561 \\
35 & 1.000 & 1.000 & 0.997 & 0.959 & 0.001 & 0.007 & 0.096 & 0.523 \\
40 & 1.000 & 1.000 & 0.997 & 0.981 & 0.001 & 0.006 & 0.083 & 0.488 \\
45 & 1.000 & 1.000 & 0.997 & 0.991 & 0.001 & 0.005 & 0.072 & 0.456 \\
\hline
\end{tabularx}
\label{table:cnn_svchosta_test}
\end{table}
\par Dapat dilihat dari tabel hasil test \ref{table:cnn_svchosta_test} diatas, semakin banyak epoch maka semakin tinggi akurasi dan semakin rendah loss nya. Lama akurasi mencapai titik tertinggi, dan loss mencapai titik terendah ditentukan oleh \textit{learning rate}.
\par Dapat dilihat berdasarkan parameter \textit{learning rate}, dimana semakin besar nilai \textit{learning rate} maka semakin cepat pula akurasi mencapai titik tertinggi, dan loss mencapai titik terendah.

% Grafik sepuluh prediksi Svchosta
\begin{figure}%[H]
\centering
\buatsubgrafik 
{public/assets/img/cnn_svchosta_train_pred01.png}
{Prediksi train lr=0.1}{.45}{}
\buatsubgrafik 
{public/assets/img/cnn_svchosta_train_pred001.png}
{Prediksi train lr=0.01}{.45}{}
\buatsubgrafik 
{public/assets/img/cnn_svchosta_train_pred0001.png}
{Prediksi train lr=0.001}{.45}{}
\buatsubgrafik 
{public/assets/img/cnn_svchosta_train_pred00001.png}
{Prediksi train lr=0.0001}{.45}{}
\buatsubgrafik 
{public/assets/img/cnn_svchosta_test_pred01.png}
{Prediksi train lr=0.1}{.45}{}
\buatsubgrafik 
{public/assets/img/cnn_svchosta_test_pred001.png}
{Prediksi test lr=0.01}{.45}{}
\buatsubgrafik 
{public/assets/img/cnn_svchosta_test_pred0001.png}
{Prediksi test lr=0.001}{.45}{}
\buatsubgrafik 
{public/assets/img/cnn_svchosta_test_pred00001.png}
{Prediksi test lr=0.0001}{.45}{}
\caption{Grafik Prediksi Svchosta CNN}
\label{fig:cnn_svchosta_pred}
\end{figure}
\par Berdasarkan grafik prediksi \ref{fig:cnn_svchosta_pred} diatas bahwa terdapat perbedaan hasil prediksi ketika parameter \textit{learning rate} diubah. Semakin kecil nilai parameter learning rate menyebabkan persebaran prediksi menjadi semakkin tinggi.
\begin{figure}%[H]
	\centering
	\includegraphics[width=0.7\textwidth]{public/assets/img/cnn_svchosta_loss.png}
	\caption{Grafik Loss Svchosta CNN}
	\label{fig:cnn_svchosta_loss}
\end{figure}
\begin{figure}%[H]
	\centering
	\includegraphics[width=0.7\textwidth]{public/assets/img/cnn_svchosta_acc.png}
	\caption{Grafik Akurasi Svchosta CNN}
	\label{fig:cnn_svchosta_acc}
\end{figure}

\par Dapat dilihat dari grafik diatas dimana parameter yang memiliki parameter \textit{learning rate} terbesar memiliki waktu konvergen yang lebih singkat dari \textit{learning rate} lainnya.
\par Waktu konvergen dapat dilihat saat kurva menuju ke satu nilai tertentu yakni nilai 1, dari grafik diatas juga dapat dilihat bahwa kurva dengan learning rate tertinggi yakni kurva berwarna biru memiliki waktu konvergen yang paling singkat dari yang lainnya.
% End Grafik Prediksi Svchosta

\subsubsection{Hasil Training CNN Rbot}
\par Berikut ini adalah tabel hasil training CNN untuk Botnet RBot
\begin{table}%[H]
\centering
\caption{Tabel Hasil Training CNN Rbot}
\begin{tabularx}{\textwidth}{|*{9}{Y|}}
\hline
    \multirow{2}{*}{Epoch} 
  & \multicolumn{4}{c|}{Akurasi}
  & \multicolumn{4}{c|}{Loss} \\
\cline{2-9}
   &      1e-1 &      1e-2 &      1e-3 &      1e-4 &      1e-1 &      1e-2 &      1e-3 &      1e-4 \\
\cline{1-9}
0  & 0.771 & 0.488 & 0.723 & 0.798 & 0.513 & 0.701 & 0.666 & 0.590 \\
5  & 1.000 & 0.998 & 0.864 & 0.803 & 0.006 & 0.056 & 0.580 & 0.561 \\
10 & 1.000 & 1.000 & 0.927 & 0.819 & 0.003 & 0.026 & 0.484 & 0.533 \\
15 & 1.000 & 1.000 & 0.950 & 0.830 & 0.002 & 0.017 & 0.388 & 0.508 \\
20 & 1.000 & 1.000 & 0.971 & 0.839 & 0.001 & 0.013 & 0.302 & 0.484 \\
25 & 1.000 & 1.000 & 0.980 & 0.862 & 0.001 & 0.011 & 0.237 & 0.464 \\
30 & 1.000 & 1.000 & 0.982 & 0.882 & 0.001 & 0.009 & 0.191 & 0.442 \\
35 & 1.000 & 1.000 & 0.984 & 0.905 & 0.001 & 0.008 & 0.157 & 0.423 \\
40 & 1.000 & 1.000 & 0.986 & 0.916 & 0.001 & 0.007 & 0.132 & 0.406 \\
45 & 1.000 & 1.000 & 0.989 & 0.923 & 0.001 & 0.006 & 0.112 & 0.387 \\
\hline
\end{tabularx}
\label{table:cnn_rbot_train}
\end{table}
\par Dari tabel hasil train \ref{table:cnn_rbot_train} diatas dapat diamati bahwa semakin banyak epoch maka semakin tinggi akurasi dan semakin rendah loss nya. Lama akurasi mencapai titik tertinggi, dan loss mencapai titik terendah ditentukan oleh \textit{learning rate}.
\par Berdasarkan parameter \textit{learning rate}, dimana semakin besar nilai \textit{learning rate} maka semakin cepat pula akurasi mencapai titik tertinggi, dan loss mencapai titik terendah.

\begin{table}%[H]
\centering
\caption{Tabel Hasil Testing CNN Rbot}
\begin{tabularx}{\textwidth}{|*{9}{Y|}}
\hline
    \multirow{2}{*}{Epoch} 
  & \multicolumn{4}{c|}{Akurasi}
  & \multicolumn{4}{c|}{Loss} \\
\cline{2-9}
   &      1e-1 &      1e-2 &      1e-3 &      1e-4 &      1e-1 &      1e-2 &      1e-3 &      1e-4 \\
\cline{1-9}
00  & 0.884 & 0.984 & 0.831 & 0.861 & 0.251 & 0.284 & 0.639 & 0.568 \\
5  & 1.000 & 1.000 & 0.900 & 0.865 & 0.008 & 0.047 & 0.549 & 0.538 \\
10 & 1.000 & 1.000 & 0.941 & 0.873 & 0.003 & 0.024 & 0.460 & 0.512 \\
15 & 1.000 & 1.000 & 0.973 & 0.881 & 0.002 & 0.015 & 0.362 & 0.489 \\
20 & 1.000 & 1.000 & 0.981 & 0.890 & 0.001 & 0.011 & 0.277 & 0.466 \\
25 & 1.000 & 1.000 & 0.986 & 0.909 & 0.001 & 0.009 & 0.215 & 0.446 \\
30 & 1.000 & 1.000 & 0.986 & 0.919 & 0.001 & 0.007 & 0.169 & 0.426 \\
35 & 1.000 & 1.000 & 0.989 & 0.933 & 0.001 & 0.006 & 0.139 & 0.407 \\
40 & 1.000 & 1.000 & 0.992 & 0.942 & 0.001 & 0.006 & 0.116 & 0.389 \\
45 & 1.000 & 1.000 & 0.992 & 0.947 & 0.001 & 0.005 & 0.097 & 0.372 \\
\hline
\end{tabularx}
\label{table:cnn_rbot_test}
\end{table}
\par Tabel hasil test \ref{table:cnn_rbot_test} diatas menunjukkan bahwa semakin banyak epoch maka semakin tinggi akurasi dan semakin rendah loss nya. Lama akurasi mencapai titik tertinggi, dan loss mencapai titik terendah ditentukan oleh \textit{learning rate}.
\par Dapat dilihat berdasarkan parameter \textit{learning rate} memiliki karakteristik dimana semakin besar nilai \textit{learning rate} maka semakin cepat pula akurasi mencapai titik tertinggi, dan loss mencapai titik terendah.
% Grafik sepuluh prediksi Rbot
\begin{figure}%[H]
\centering
\buatsubgrafik 
{public/assets/img/cnn_rbot_train_pred01.png}
{Prediksi train lr=0.1}{.45}{}
\buatsubgrafik 
{public/assets/img/cnn_rbot_train_pred001.png}
{Prediksi train lr=0.01}{.45}{}
\buatsubgrafik 
{public/assets/img/cnn_rbot_train_pred0001.png}
{Prediksi train lr=0.001}{.45}{}
\buatsubgrafik 
{public/assets/img/cnn_rbot_train_pred00001.png}
{Prediksi train lr=0.0001}{.45}{}
\buatsubgrafik 
{public/assets/img/cnn_rbot_test_pred01.png}
{Prediksi train lr=0.1}{.45}{}
\buatsubgrafik 
{public/assets/img/cnn_rbot_test_pred001.png}
{Prediksi test lr=0.01}{.45}{}
\buatsubgrafik 
{public/assets/img/cnn_rbot_test_pred0001.png}
{Prediksi test lr=0.001}{.45}{}
\buatsubgrafik 
{public/assets/img/cnn_rbot_test_pred00001.png}
{Prediksi test lr=0.0001}{.45}{}
\caption{Grafik Prediksi Rbot CNN}
\label{fig:cnn_rbot_pred}
\end{figure}
\par Dapat dilihat dari grafik prediksi \ref{fig:cnn_rbot_pred} diatas bahwa terdapat perbedaan hasil prediksi ketika parameter \textit{learning rate} diubah. Semakin kecil nilai parameter learning rate menyebabkan persebaran prediksi menjadi semakkin tinggi.
\begin{figure}%[H]
	\includegraphics[width=0.7\textwidth]{public/assets/img/cnn_rbot_loss.png}
	\caption{Grafik Loss Rbot CNN}
	\label{fig:cnn_rbot_loss}
\end{figure}
\begin{figure}%[H]
	\includegraphics[width=0.7\textwidth]{public/assets/img/cnn_rbot_acc.png}
	\caption{Grafik Akurasi Rbot CNN}
	\label{fig:cnn_rbot_acc}
\end{figure}
\par Grafik akurasi, dan loss menunjukkan bahwa parameter \textit{learning rate} terbesar memiliki waktu konvergen yang lebih singkat dari \textit{learning rate} lainnya. Dapat diamati juga sama seperti pola data hasil training sebelumnya bahwa hasil training paling konvergen diperoleh pada learning rate terendah yakni 0.1.
% End Grafik Prediksi Rbot

\subsubsection{Hasil Training CNN Neris}
\begin{table}%[H]
\centering
\caption{Tabel Hasil Training CNN Neris}
\begin{tabularx}{\textwidth}{|*{9}{Y|}}
\hline
    \multirow{2}{*}{Epoch} 
  & \multicolumn{4}{c|}{Akurasi}
  & \multicolumn{4}{c|}{Loss} \\
\cline{2-9}
   &      1e-1 &      1e-2 &      1e-3 &      1e-4 &      1e-1 &      1e-2 &      1e-3 &      1e-4 \\
\cline{1-9}
0  & 0.885 & 0.848 & 0.749 & 0.460 & 0.335 & 0.435 & 0.570 & 0.669 \\
5  & 0.926 & 0.968 & 0.926 & 0.875 & 0.232 & 0.135 & 0.328 & 0.558 \\
10 & 0.933 & 0.982 & 0.927 & 0.926 & 0.223 & 0.076 & 0.260 & 0.467 \\
15 & 0.937 & 0.983 & 0.935 & 0.923 & 0.211 & 0.061 & 0.225 & 0.445 \\
20 & 0.935 & 0.989 & 0.930 & 0.927 & 0.197 & 0.059 & 0.196 & 0.421 \\
25 & 0.939 & 0.987 & 0.934 & 0.923 & 0.199 & 0.058 & 0.179 & 0.408 \\
30 & 0.943 & 0.992 & 0.935 & 0.925 & 0.166 & 0.049 & 0.162 & 0.390 \\
35 & 0.933 & 0.992 & 0.943 & 0.929 & 0.212 & 0.046 & 0.146 & 0.377 \\
40 & 0.945 & 0.992 & 0.952 & 0.927 & 0.191 & 0.046 & 0.135 & 0.364 \\
45 & 0.934 & 0.992 & 0.958 & 0.926 & 0.188 & 0.045 & 0.128 & 0.353 \\
\hline
\end{tabularx}
\label{table:cnn_neris_train}
\end{table}
\par Dapat dilihat dari tabel \ref{table:cnn_neris_train} hasil train diatas, semakin banyak epoch maka semakin tinggi akurasi dan semakin rendah loss nya. Lama akurasi mencapai titik tertinggi, dan loss mencapai titik terendah ditentukan oleh \textit{learning rate}.
\par Berdasarkan parameter \textit{learning rate}, dapat diamati bahwa semakin besar nilai \textit{learning rate} maka semakin cepat pula akurasi mencapai titik tertinggi, dan loss mencapai titik terendah.

\begin{table}%[H]
\centering
\caption{Tabel Hasil Testing CNN Neris}
\begin{tabularx}{\textwidth}{|*{9}{Y|}}
\hline
    \multirow{2}{*}{Epoch} 
  & \multicolumn{4}{c|}{Akurasi}
  & \multicolumn{4}{c|}{Loss} \\
\cline{2-9}
   &      1e-1 &      1e-2 &      1e-3 &      1e-4 &      1e-1 &      1e-2 &      1e-3 &      1e-4 \\
\cline{1-9}
0  & 0.937 & 0.915 & 0.933 & 0.570 & 0.298 & 0.294 & 0.418 & 0.610 \\
5  & 0.576 & 0.577 & 0.940 & 0.939 & 0.855 & 0.344 & 0.282 & 0.480 \\
10 & 0.939 & 0.983 & 0.939 & 0.939 & 0.211 & 0.074 & 0.241 & 0.421 \\
15 & 0.942 & 0.990 & 0.940 & 0.938 & 0.209 & 0.284 & 0.206 & 0.393 \\
20 & 0.949 & 0.985 & 0.946 & 0.940 & 0.288 & 0.051 & 0.182 & 0.374 \\
25 & 0.941 & 0.993 & 0.946 & 0.940 & 0.188 & 0.057 & 0.162 & 0.359 \\
30 & 0.931 & 0.993 & 0.947 & 0.941 & 0.203 & 0.042 & 0.148 & 0.345 \\
35 & 0.933 & 0.993 & 0.947 & 0.941 & 0.208 & 0.041 & 0.138 & 0.333 \\
40 & 0.576 & 0.993 & 0.953 & 0.941 & 1.248 & 0.039 & 0.122 & 0.321 \\
45 & 0.950 & 0.993 & 0.969 & 0.941 & 0.209 & 0.040 & 0.123 & 0.310 \\
\hline
\end{tabularx}
\label{table:cnn_neris_test}
\end{table}
\par Dapat dilihat dari tabel hasil test \ref{table:cnn_neris_test} diatas, semakin banyak epoch maka semakin tinggi akurasi dan semakin rendah loss nya. Lama akurasi mencapai titik tertinggi, dan loss mencapai titik terendah ditentukan oleh \textit{learning rate}.
\par Dapat dilihat berdasarkan parameter \textit{learning rate}, dimana semakin besar nilai \textit{learning rate} maka semakin cepat pula akurasi mencapai titik tertinggi, dan loss mencapai titik terendah.

% Grafik sepuluh prediksi Neris
\begin{figure}%[H]
\centering
\buatsubgrafik 
{public/assets/img/cnn_neris_train_pred01.png}
{Prediksi train lr=0.1}{.45}{}
\buatsubgrafik 
{public/assets/img/cnn_neris_train_pred001.png}
{Prediksi train lr=0.01}{.45}{}
\buatsubgrafik 
{public/assets/img/cnn_neris_train_pred0001.png}
{Prediksi train lr=0.001}{.45}{}
\buatsubgrafik 
{public/assets/img/cnn_neris_train_pred00001.png}
{Prediksi train lr=0.0001}{.45}{}
\buatsubgrafik 
{public/assets/img/cnn_neris_test_pred01.png}
{Prediksi train lr=0.1}{.45}{}
\buatsubgrafik 
{public/assets/img/cnn_neris_test_pred001.png}
{Prediksi test lr=0.01}{.45}{}
\buatsubgrafik 
{public/assets/img/cnn_neris_test_pred0001.png}
{Prediksi test lr=0.001}{.45}{}
\buatsubgrafik 
{public/assets/img/cnn_neris_test_pred00001.png}
{Prediksi test lr=0.0001}{.45}{}
\caption{Grafik Prediksi Neris CNN}
\label{fig:cnn_neris_pred}
\end{figure}
\par Sama seperti sebelumnya dari grafik prediksi \ref{fig:cnn_neris_pred} diatas bahwa terdapat perbedaan hasil prediksi ketika parameter \textit{learning rate} diubah. Semakin kecil nilai parameter learning rate menyebabkan persebaran prediksi menjadi semakkin tinggi.
\begin{figure}%[H]
	\centering
	\includegraphics[width=0.7\textwidth]{public/assets/img/cnn_neris_loss.png}
	\caption{Grafik Loss Neris CNN}
	\label{fig:cnn_neris_loss}
\end{figure}
\begin{figure}%[H]
	\centering
	\includegraphics[width=0.7\textwidth]{public/assets/img/cnn_neris_acc.png}
	\caption{Grafik Akurasi Neris CNN}
	\label{fig:cnn_neris_acc}
\end{figure}
\par Grafik akurasi, dan loss diatas menunjukkan bahwa parameter yang memiliki \textit{learning rate} terbesar memiliki waktu konvergen yang lebih singkat dari \textit{learning rate} lainnya. Data yang diperoleh juga sama seperti data hasil training sebelumnya, dimana data yang paling konvergen merupakan data dengan \textit{learning rate} terendah yakni 0.1.
% End Grafik Prediksi Neris

\subsubsection{Hasil Training CNN Single}
\begin{table}%[H]
\centering
\caption{Tabel Hasil Training CNN Single}
\begin{tabularx}{\textwidth}{|*{9}{Y|}}
\hline
    \multirow{2}{*}{Epoch} 
  & \multicolumn{4}{c|}{Akurasi}
  & \multicolumn{4}{c|}{Loss} \\
\cline{2-9}
   &      1e-1 &      1e-2 &      1e-3 &      1e-4 &      1e-1 &      1e-2 &      1e-3 &      1e-4 \\
\cline{1-9}
0  &  0.835 &  0.472 &  0.659 &  0.102 &  0.397 &  0.681 &  0.664 &  1.022 \\
5  &  0.948 &  0.942 &  0.879 &  0.152 &  0.098 &  0.170 &  0.386 &  0.866 \\
10 &  0.942 &  0.950 &  0.948 &  0.365 &  0.082 &  0.105 &  0.279 &  0.756 \\
15 &  0.948 &  0.945 &  0.950 &  0.619 &  0.074 &  0.094 &  0.222 &  0.668 \\
20 &  0.945 &  0.950 &  0.961 &  0.772 &  0.074 &  0.086 &  0.186 &  0.597 \\
25 &  0.940 &  0.950 &  0.955 &  0.843 &  0.069 &  0.082 &  0.162 &  0.540 \\
30 &  0.945 &  0.940 &  0.955 &  0.877 &  0.068 &  0.080 &  0.147 &  0.495 \\
35 &  0.948 &  0.950 &  0.955 &  0.906 &  0.067 &  0.079 &  0.134 &  0.454 \\
40 &  0.942 &  0.950 &  0.948 &  0.927 &  0.066 &  0.077 &  0.126 &  0.421 \\
45 &  0.945 &  0.945 &  0.945 &  0.937 &  0.066 &  0.075 &  0.120 &  0.392 \\
\hline
\end{tabularx}
\label{table:cnn_single_train}
\end{table}

\par Dapat diamati pada tabel hasil train \ref{table:cnn_single_train} diatas, semakin banyak epoch maka semakin tinggi akurasi dan semakin rendah loss nya. Lama akurasi mencapai titik tertinggi, dan loss mencapai titik terendah ditentukan oleh \textit{learning rate}.

\par Berdasarkan parameter \textit{learning rate}, dapat diamati bahwa semakin besar nilai \textit{learning rate} maka semakin cepat pula akurasi mencapai titik tertinggi, dan loss mencapai titik terendah.

\begin{table}%[H]
\centering
\caption{Tabel Hasil Testing CNN Single}
\begin{tabularx}{\textwidth}{|*{9}{Y|}}
\hline
    \multirow{2}{*}{Epoch} 
  & \multicolumn{4}{c|}{Akurasi}
  & \multicolumn{4}{c|}{Loss} \\
\cline{2-9}
   &      1e-1 &      1e-2 &      1e-3 &      1e-4 &      1e-1 &      1e-2 &      1e-3 &      1e-4 \\
\cline{1-9}
0  &  0.950 &  0.820 &  0.843 &  0.067 &  0.147 &  0.466 &  0.593 &  1.031 \\
5  &  0.971 &  0.967 &  0.919 &  0.102 &  0.063 &  0.106 &  0.344 &  0.878 \\
10 &  0.971 &  0.971 &  0.967 &  0.356 &  0.054 &  0.074 &  0.234 &  0.755 \\
15 &  0.972 &  0.971 &  0.971 &  0.686 &  0.048 &  0.063 &  0.177 &  0.657 \\
20 &  0.972 &  0.971 &  0.971 &  0.831 &  0.046 &  0.061 &  0.147 &  0.580 \\
25 &  0.972 &  0.971 &  0.971 &  0.884 &  0.045 &  0.055 &  0.129 &  0.515 \\
30 &  0.972 &  0.971 &  0.971 &  0.914 &  0.045 &  0.053 &  0.114 &  0.465 \\
35 &  0.972 &  0.971 &  0.971 &  0.936 &  0.044 &  0.052 &  0.104 &  0.422 \\
40 &  0.972 &  0.972 &  0.971 &  0.950 &  0.043 &  0.052 &  0.097 &  0.388 \\
45 &  0.972 &  0.972 &  0.971 &  0.960 &  0.043 &  0.050 &  0.091 &  0.359 \\
\hline
\end{tabularx}
\label{table:cnn_single_test}
\end{table}
\par Dapat dilihat dari tabel hasil test \ref{table:cnn_single_test} diatas, semakin banyak epoch maka semakin tinggi akurasi dan semakin rendah loss nya. Lama akurasi mencapai titik tertinggi, dan loss mencapai titik terendah ditentukan oleh \textit{learning rate}.

\par Dapat diamati berdasarkan parameter \textit{learning rate}, memiliki karakteristik dimana semakin besar nilai \textit{learning rate} maka semakin cepat pula akurasi mencapai titik tertinggi, dan loss mencapai titik terendah.

% Grafik sepuluh prediksi Single
\begin{figure}%[H]
\centering
\buatsubgrafik 
{public/assets/img/cnn_single_train_pred01.png}
{Prediksi train lr=0.1}{.45}{}
\buatsubgrafik 
{public/assets/img/cnn_single_train_pred001.png}
{Prediksi train lr=0.01}{.45}{}
\buatsubgrafik 
{public/assets/img/cnn_single_train_pred0001.png}
{Prediksi train lr=0.001}{.45}{}
\buatsubgrafik 
{public/assets/img/cnn_single_train_pred00001.png}
{Prediksi train lr=0.0001}{.45}{}
\buatsubgrafik 
{public/assets/img/cnn_single_test_pred01.png}
{Prediksi train lr=0.1}{.45}{}
\buatsubgrafik 
{public/assets/img/cnn_single_test_pred001.png}
{Prediksi test lr=0.01}{.45}{}
\buatsubgrafik 
{public/assets/img/cnn_single_test_pred0001.png}
{Prediksi test lr=0.001}{.45}{}
\buatsubgrafik 
{public/assets/img/cnn_single_test_pred00001.png}
{Prediksi test lr=0.0001}{.45}{}
\caption{Grafik Prediksi Single CNN}
\label{fig:cnn_single_pred}
\end{figure}
\par Dapat dilihat dari grafik prediksi \ref{fig:cnn_single_pred} diatas bahwa terdapat perbedaan hasil prediksi ketika parameter \textit{learning rate} diubah. Semakin kecil nilai parameter learning rate menyebabkan persebaran prediksi menjadi semakkin tinggi.
\begin{figure}%[H]
	\centering
	\includegraphics[width=0.7\textwidth]{public/assets/img/cnn_single_loss.png}
	\caption{Grafik Loss Single CNN}
	\label{fig:cnn_single_loss}
\end{figure}
\begin{figure}%[H]
	\centering
	\includegraphics[width=0.7\textwidth]{public/assets/img/cnn_single_acc.png}
	\caption{Grafik Akurasi Single CNN}
	\label{fig:cnn_single_acc}
\end{figure}
\par Dari grafik akurasi, dan loss dapat diamati bahwa parameter dengan \textit{learning rate} terbesar memiliki waktu konvergen yang lebih singkat dari \textit{learning rate} lainnya. Sama seperti data hasil training sebelumnya, data dengan konvergensi tertinggi ada pada \textit{learning rate} 0.1.
% End Grafik Prediksi Single

\subsubsection{Hasil Training CNN Multiple}
\begin{table}%[H]
\centering
\caption{Tabel Hasil Training CNN Multiple}
\begin{tabularx}{\textwidth}{|*{9}{Y|}}
\hline
    \multirow{2}{*}{Epoch} 
  & \multicolumn{4}{c|}{Akurasi}
  & \multicolumn{4}{c|}{Loss} \\
\cline{2-9}
   &      1e-1 &      1e-2 &      1e-3 &      1e-4 &      1e-1 &      1e-2 &      1e-3 &      1e-4 \\
\cline{1-9}
0  &  0.698 &  0.589 &  0.310 &  0.304 &  0.595 &  0.654 &  0.805 &  1.053 \\
5  &  0.771 &  0.779 &  0.729 &  0.304 &  0.326 &  0.337 &  0.604 &  0.950 \\
10 &  0.792 &  0.753 &  0.792 &  0.336 &  0.268 &  0.293 &  0.503 &  0.866 \\
15 &  0.792 &  0.775 &  0.794 &  0.496 &  0.261 &  0.278 &  0.442 &  0.797 \\
20 &  0.783 &  0.771 &  0.806 &  0.601 &  0.256 &  0.272 &  0.395 &  0.745 \\
25 &  0.798 &  0.775 &  0.787 &  0.638 &  0.254 &  0.267 &  0.364 &  0.704 \\
30 &  0.808 &  0.785 &  0.791 &  0.670 &  0.250 &  0.262 &  0.345 &  0.672 \\
35 &  0.812 &  0.743 &  0.785 &  0.684 &  0.244 &  0.260 &  0.330 &  0.644 \\
40 &  0.785 &  0.791 &  0.785 &  0.700 &  0.243 &  0.256 &  0.320 &  0.623 \\
45 &  0.794 &  0.787 &  0.802 &  0.717 &  0.242 &  0.254 &  0.312 &  0.598 \\
\hline
\end{tabularx}
\label{table:cnn_multi_train}
\end{table}
\par Berdasarkan dari tabel hasil train \ref{table:cnn_multi_train} diatas, semakin banyak epoch maka semakin tinggi akurasi dan semakin rendah loss nya. Lama akurasi mencapai titik tertinggi, dan loss mencapai titik terendah ditentukan oleh \textit{learning rate}.
\par Dapat dilihat parameter \textit{learning rate} menunjukkan bahwa semakin besar nilai \textit{learning rate} maka semakin cepat pula akurasi mencapai titik tertinggi, dan loss mencapai titik terendah.

\begin{table}%[H]
\centering
\caption{Tabel Hasil Testing CNN Multiple}
\begin{tabularx}{\textwidth}{|*{9}{Y|}}
\hline
    \multirow{2}{*}{Epoch} 
  & \multicolumn{4}{c|}{Akurasi}
  & \multicolumn{4}{c|}{Loss} \\
\cline{2-9}
   &      1e-1 &      1e-2 &      1e-3 &      1e-4 &      1e-1 &      1e-2 &      1e-3 &      1e-4 \\
\cline{1-9}
0  &  0.764 &  0.797 &  0.241 &  0.219 &  0.429 &  0.471 &  0.790 &  1.119 \\
5  &  0.855 &  0.864 &  0.797 &  0.219 &  0.272 &  0.250 &  0.562 &  1.024 \\
10 &  0.868 &  0.871 &  0.847 &  0.222 &  0.206 &  0.215 &  0.439 &  0.922 \\
15 &  0.871 &  0.874 &  0.857 &  0.375 &  0.188 &  0.203 &  0.368 &  0.818 \\
20 &  0.876 &  0.874 &  0.857 &  0.621 &  0.185 &  0.198 &  0.322 &  0.735 \\
25 &  0.876 &  0.871 &  0.862 &  0.716 &  0.181 &  0.192 &  0.292 &  0.675 \\
30 &  0.874 &  0.871 &  0.865 &  0.740 &  0.179 &  0.189 &  0.270 &  0.631 \\
35 &  0.876 &  0.876 &  0.866 &  0.747 &  0.175 &  0.186 &  0.255 &  0.598 \\
40 &  0.876 &  0.874 &  0.865 &  0.760 &  0.175 &  0.186 &  0.247 &  0.570 \\
45 &  0.876 &  0.874 &  0.869 &  0.786 &  0.174 &  0.184 &  0.236 &  0.546 \\
\hline
\end{tabularx}
\label{table:cnn_multi_test}
\end{table}
\par Tabel hasil test \ref{table:cnn_multi_test} diatas menunjukkansemakin banyak epoch maka semakin tinggi akurasi dan semakin rendah loss nya. Lama akurasi mencapai titik tertinggi, dan loss mencapai titik terendah ditentukan oleh \textit{learning rate}.
\par Berdasarkan parameter \textit{learning rate} menunjukkan bahwa semakin besar nilai \textit{learning rate} maka semakin cepat pula akurasi mencapai titik tertinggi, dan loss mencapai titik terendah.
% Grafik sepuluh prediksi Multiple
\begin{figure}%[H]
\centering
\buatsubgrafik 
{public/assets/img/cnn_multi_train_pred01.png}
{Prediksi train lr=0.1}{.45}{}
\buatsubgrafik 
{public/assets/img/cnn_multi_train_pred001.png}
{Prediksi train lr=0.01}{.45}{}
\buatsubgrafik 
{public/assets/img/cnn_multi_train_pred0001.png}
{Prediksi train lr=0.001}{.45}{}
\buatsubgrafik 
{public/assets/img/cnn_multi_train_pred00001.png}
{Prediksi train lr=0.0001}{.45}{}
\buatsubgrafik 
{public/assets/img/cnn_multi_test_pred01.png}
{Prediksi train lr=0.1}{.45}{}
\buatsubgrafik 
{public/assets/img/cnn_multi_test_pred001.png}
{Prediksi test lr=0.01}{.45}{}
\buatsubgrafik 
{public/assets/img/cnn_multi_test_pred0001.png}
{Prediksi test lr=0.001}{.45}{}
\buatsubgrafik 
{public/assets/img/cnn_multi_test_pred00001.png}
{Prediksi test lr=0.0001}{.45}{}
\caption{Grafik Prediksi Multiple CNN}
\label{fig:cnn_multi_pred}
\end{figure}
\par Dapat dilihat dari grafik prediksi \ref{fig:cnn_multi_pred} diatas bahwa terdapat perbedaan hasil prediksi ketika parameter \textit{learning rate} diubah. Semakin kecil nilai parameter learning rate menyebabkan persebaran prediksi menjadi semakkin tinggi.
\begin{figure}%[H]
	\centering
	\includegraphics[width=0.7\textwidth]{public/assets/img/cnn_multi_loss.png}
	\caption{Grafik Loss Multiple CNN}
	\label{fig:cnn_multi_loss}
\end{figure}
\begin{figure}%[H]
	\centering
	\includegraphics[width=0.7\textwidth]{public/assets/img/cnn_multi_acc.png}
	\caption{Grafik Akurasi Multiple CNN}
	\label{cnn:cnn_multi_acc}
\end{figure}
\par Dapat dilihat dari grafik akurasi, dan loss dimana data yang memiliki parameter \textit{learning rate} terbesar memiliki waktu konvergen yang lebih singkat dari \textit{learning rate} lainnya.
\par Semakin kecil \textit{learning rate} nya, maka semakin lama waktu konvergensi nya. Dari grafik diatas diperoleh waktu konvergensi tercepat ada pada \textit{learning rate} 0.1.
% END TABEL
\subsection{Data Hasil Detection Rate}
\par Berikut ini adalah data hasil yang diperoleh dari kinerja snort dan keras dalam proses pendeteksian paket yang dipresentasikan dalam bentuk detection rate.
\begin{table}%[H]
\centering
\begin{tabularx}{\textwidth}{|*{9}{Y|}}
\hline
 \multirow{2}{*}{Ukuran}
  & \multicolumn{3}{c|}{Runtime}
  & \multirow{2}{*}{Detection} \\
\cline{2-4}
 Paket & Snort (s) & LSTM (s) & CNN (s) & Rate (p/s)\\
\hline
100 & 1.170 & 0.877 & 4.445  & 14.221\\
200 & 1.473 & 0.563 & 20.068 &  9.048\\
300 & 1.598 & 0.549 & 29.517 &  9.474\\
400 & 1.641 & 0.576 & 32.629 & 11.479\\
500 & 1.308 & 0.461 & 40.888 & 11.721\\
600 & 1.370 & 0.433 & 49.999 & 11.582\\
700 & 1.418 & 0.463 & 61.193 & 11.098\\
800 & 1.669 & 0.488 & 88.764 &  8.798\\
900 & 1.210 & 0.501 & 68.418 & 12.833\\
1000& 1.857 & 0.526 & 82.615 & 11.764\\
\hline
\end{tabularx}
\caption{Data hasil detection rate preprocessor}
\label{table:detection_rate}
\end{table}
\par Berdasarkan tabel \ref{table:detection_rate} dapat diamati bahwa \textit{detection rate} pada sisi snort, dan LSTM memiliki variasi yang fluktuatif. Pada sisi CNN \textit{detection rate} memiliki pola dimana semakin besar ukuran paket, maka semakin lama pula proses pendeteksian pada sisi CNN nya. Untuk total detection rate diperhitungkan terhadap ukuran paket per total waktu pada Snort CNN dan LSTM, diperoleh Rate yang fluktuatif.
\end{document}
